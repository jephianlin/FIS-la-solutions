\section{The Rational Canonical Form}
\begin{enumerate}
\item \begin{enumerate}
\item Yes. See Theorem 7.17.
\item No. Let $T(a,b)=(a,2b)$ be a transformation. Then the basis $\beta=\{(1,1),(1,2)\}$ is a $T$-cyclic basis generated by $(1,1)$. But it is not a rational canonical basis.
\item No. See Theorem 7.22.
\item Yes. If $A$ is a square matrix with its rational canonical form $C$ with rational canonical basis $\beta$, then we have $C=[L_A]_{\beta}$.
\item Yes. See Theorem 7.23(a).
\item No. They are in general diffrent. For example, the dot diagram of the matrix $\begin{pmatrix}1&1\\0&1\end{pmatrix}$ has only one dot. It could not forms a basis.
\item Yes. The matrix is similar to its Jordan canonical form and its rational canonical form. Hence the two forms should be similar.
\end{enumerate}
\item Find the factorization of the \charpoly{}. Find the basis of $K_{\phi}$ for each some monic irreducibla polyomial factor consisting of $T$-cyclic bases through the proof of Theorem 7.22. Write down the basis with some appropriate order as the columns of $Q$. Then compute $C=Q^{-1}AQ$ or find $C$ by the dot diagram. And I want to emphasize that I compute these answers in Exercises 7.4.2 and 7.4.3 by HAND! 
\begin{enumerate}
\item It is a Jordan canonical form. So 
\[Q=\begin{pmatrix}0&0&1\\0&1&6\\1&3&9\end{pmatrix}\]
and 
\[C=\begin{pmatrix}0&0&27\\1&0&-27\\0&1&9\end{pmatrix}.\]
\item It has been already the rational canonical form since the \charpoly{} $t^2+t+1$ is irreducible in $\R$. So $C=A$ and $Q=I$.
\item It is diagonalizable in $\C$. So 
\[Q=\begin{pmatrix}1 & 1\cr \frac{\sqrt{3}\,i+1}{2} & \frac{1-\sqrt{3}\,i}{2}\end{pmatrix}\]
and 
\[C=\begin{pmatrix}-\frac{\sqrt{3}\,i+1}{2} & 0\cr 0 & \frac{\sqrt{3}\,i-1}{2}\end{pmatrix}.\]
\item Try the generating vector $(1,0,0,0)$. So 
\[Q=\begin{pmatrix}1 & 0 & -7 & -4\cr 0 & 1 & -4 & -3\cr 0 & 0 & -4 & -4\cr 0 & 0 & -4 & -8\end{pmatrix}\]
and 
\[C=\begin{pmatrix}0 & 0 & 0 & -1\cr 1 & 0 & 0 & 0\cr 0 & 1 & 0 & -2\cr 0 & 0 & 1 & 0\end{pmatrix}.\]
\item Use $(0,-1,0,1)$ and $(3,1,1,0)$ as generating vectors. So
\[Q=\begin{pmatrix}0 & -3 & 3 & 8\cr -1 & -2 & 1 & 5\cr 0 & -3 & 1 & 5\cr 1 & -4 & 0 & 7\end{pmatrix}\]
and 
\[C=\begin{pmatrix}0 & -2 & 0 & 0\cr 1 & 0 & 0 & 0\cr 0 & 0 & 0 & -3\cr 0 & 0 & 1 & 0\end{pmatrix}.\]
\end{enumerate}
\item Write down the matrix representation $A$ by some basis $\beta$ and find the rational $C=Q^{-1}AQ$ for some inveritble $Q$. Then the rational canonical basis is the basis corresponding the columns of $Q$.\begin{enumerate}
\item Let 
\[\beta=\{1,x,x^2,x^3\}.\]
Then 
\[Q=\begin{pmatrix}1 & 0 & 0 & 0\cr 0 & 1 & 3 & 0\cr 0 & 0 & 0 & 3\cr 0 & 0 & -1 & -2\end{pmatrix}\]
and 
\[C=\begin{pmatrix}0 & -1 & 0 & 0\cr 1 & 0 & 0 & 0\cr 0 & 0 & 0 & 0\cr 0 & 0 & 0 & 0\end{pmatrix}.\]
\item Let $\beta=S$. Then 
\[Q=\begin{pmatrix}1 & 0 & 0 & 0\cr 0 & 1 & 3 & 0\cr 0 & 0 & 0 & 3\cr 0 & 0 & -1 & -2\end{pmatrix}\]
and 
\[C=\begin{pmatrix}0 & 0 & 0 & -1\cr 1 & 0 & 0 & 0\cr 0 & 1 & 0 & -2\cr 0 & 0 & 1 & 0\end{pmatrix}.\]
\item Let 
\[\beta=\{\begin{pmatrix}1&0\\0&0\end{pmatrix},\begin{pmatrix}0&1\\0&0\end{pmatrix},\begin{pmatrix}0&0\\1&0\end{pmatrix},\begin{pmatrix}0&0\\0&1\end{pmatrix}\}.\]
Then 
\[Q=\begin{pmatrix}1 & 0 & 0 & 0\cr 0 & 0 & 1 & 0\cr 0 & -1 & 0 & 0\cr 0 & 0 & 0 & -1\end{pmatrix}\]
and 
\[C=\begin{pmatrix}0 & -1 & 0 & 0\cr 1 & 1 & 0 & 0\cr 0 & 0 & 0 & -1\cr 0 & 0 & 1 & 1\end{pmatrix}.\]
\item Let 
\[\beta=\{\begin{pmatrix}1&0\\0&0\end{pmatrix},\begin{pmatrix}0&1\\0&0\end{pmatrix},\begin{pmatrix}0&0\\1&0\end{pmatrix},\begin{pmatrix}0&0\\0&1\end{pmatrix}\}.\]
Then 
\[Q=\begin{pmatrix}1 & 0 & 0 & 0\cr 0 & 0 & 1 & 0\cr 0 & -1 & 0 & 0\cr 0 & 0 & 0 & -1\end{pmatrix}\]
and 
\[C=\begin{pmatrix}0 & -1 & 0 & 0\cr 1 & 1 & 0 & 0\cr 0 & 0 & 0 & -1\cr 0 & 0 & 1 & 1\end{pmatrix}.\]
\item Let $\beta=S$. Then 
\[Q=\begin{pmatrix}0 & -2 & 1 & 0\cr 1 & 0 & 0 & 1\cr 1 & 0 & 0 & -1\cr 0 & 2 & 1 & 0\end{pmatrix}\]
and 
\[C=\begin{pmatrix}0 & -4 & 0 & 0\cr 1 & 0 & 0 & 0\cr 0 & 0 & 0 & 0\cr 0 & 0 & 0 & 0\end{pmatrix}.\]
\end{enumerate}
\item \begin{enumerate}
\item We may write an element in $R(\phi(T))$ as $\phi(T)(x)$ for some $x$. Since $(\phi(T))^m(v)=0$ for all $v$, we have $(\phi(T))^{m-1}(\phi(T)(x))=(\phi(T))^m(x)=0$.
\item The matrix 
\[\begin{pmatrix}1&1&0\\0&1&0\\0&0&1\end{pmatrix}\]
has minimal polynomial $(t-1)^2$. Compute $R(L_A-I)=\sp\{(1,0,0)\}$. But $(0,0,1)$ is an element in $N(L_A-I)$ but not in $R(L_A-I)$.
\item We know that the minimal polynomial $p(t)$ of the restriction of $T$ divides $(\phi(t))^m$ by Exercise 7.3.10. Pick an element $x$ such that $(\phi(T))^{m-1}(x)\neq 0$. Then we know that $y=\phi(T)(x)$ is an element in $R(\phi(T))$ and $(\phi(T))^{m-2}(y)\neq 0$. Hence $p(t)$ must be $(\phi(t))^{m-1}$.
\end{enumerate}
\item If the rational canonical form of $T$ is a diagonal matrix, then $T$ is diagonalizable naturally. Conversely, if $T$ is diagonalizable, then the \charpoly{} of $T$ splits and $E_{\lambda}=K_{\phi_{\lambda}}$, where $\phi_{\lambda}=t-\lambda$, for each eigenvalue $\lambda$. This means each cyclic basis in $K_{\phi_{\lambda}}$ is of size $1$. That is, a rational canonical basis consisting of eigenvectors. So the rational canonical form of $T$ is a diagonal matrix.
\item Here we denote the degree of $\phi_1$ and $\phi_2$ by $a$ and $b$ respectly. \begin{enumerate}
\item By Theorem 7.23(b) we know the dimension of $K_{\phi_1}$ and $K_{\phi_2}$ are $a$ and $b$ respectly. Pick a nonzero element $v_1$ in $K_{\phi_1}$. The $T$-annihilator of $v_1$ divides $\phi_1$. Hence the $T$-annihilator of $v_1$ is $\phi_1$. Find the nonzero vector $v_2$ in $K_{\phi_2}$ similarly such that the $T$-annihilator of $v_2$ is $\phi_2$. Thus $\beta_{v_1}\cup \beta_{v_2}$ is a basis of $V$ by Theorem 7.19 and the fact that $|\beta_{v_1}\cup \beta_{v_2}|=a+b=n$.
\item Pick $v_3=v_1+v_2$, where $v_1$ and $v_2$ are the two vectors given in the previous question. Since $\phi_1(T)(v_2)\neq 0$ and $\phi_2(T)(v_1)\neq 0$ by Theorem 7.18. The $T$-annihilator of $v_3$ cannot be $\phi_1$ and $\phi_2$. So the final possibility of the $T$-annihilator is $\phi_1\phi_2$.
\item The first one has two blocks but the second one has only one block.
\end{enumerate}
\item By the definition of $m_i$, we know the 
\[N(\phi_i(T)^{m_i-1}\neq N(\phi_i(T)^{m_i}=N(\phi_i(T)^{m_i+1}=K_{\phi_i}.\]
Apply Theorem 7.24 and get the result.
\item If $\phi(T)$ is not injective, we can find a nonzero element $x$ such that $\phi(T)(x)=0$. Hence the $T$-annihilator $p(t)$ divides $\phi(t)$ by Exercise 7.3.15(b). This means $p(t)=\phi(t)$. If $f(t)$ is the \charpoly{} of $T$, then we have $f(T)(x)=0$ by Cayley-Hamilton Theorem. Again by Exercise 7.3.15(b) we have $\phi(t)$ divides $f(t)$.
\item Since the disjoint union of $\beta_i$'s is a basis, each $\beta_i$ is independent and forms a basis of $\sp(\beta_i)$. Now denote 
\[W_i=\sp(\gamma_i)=\sp(\beta_i).\]
Thus $V=\oplus_iW_i$ by Theorem 5.10. And the set $\gamma=\cup_i\gamma_i$ is a basis by Exercise 1.6.33.
\item Since $x\in C_y$, we may assume $x=T^m(y)$ for some integer $m$. If $\phi(t)$ is the $T$-annihilator of $x$ and $p(t)$ is the $T$-annihilator of $y$, then we know $p(T)(x)=p(T)(T^m(y))=T^m(p(T)(y))=0$. Hence $p(t)$ is a factor of $\phi(t)$. If $x=0$, then we have $p(t)=1$ and $y=0$. The statement is true for this case. So we assume that $x\neq 0$. Thus we know that $y\neq 0$ otherwise $x$ is zero. Hence we know $p(t)=\phi(t)$. By Exercise 7.3.15, the dimension of $C_x$ is equal to the dimension of $C_y$. Since $x=T^m(y)$ we know that $C_x\subset C_y$. Finally we know that they are the same since they have the same dimension.
\item \begin{enumerate}
\item Since the rational canonical basis exists, we get the result by Theorem 5.10.
\item This comes from Theorem 5.25.
\end{enumerate}
\item Let $\beta=\cup_iC_{v_i}$. The statement holds by Theorem 5.10.
\end{enumerate}