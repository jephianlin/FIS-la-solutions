\section{Inner Products and Norms}
\begin{enumerate}
\item \begin{enumerate}
\item Yes. It's the definition.
\item Yes. See the paragraph at the beginning of this chapter.
\item No. It's conjugate linear in the second component. For example, in $\C$ with standard inner product function we have $\lag i,i\rag =1$ but not $i\lag i,1\rag =-1$.
\item No. We may define the inner product function $f$ on $\R$ to be $f(u,v)=2uv$.
\item No. Theorem 6.2 does not assume that the dimension should be finite.
\item No. We may define the conjugate-transpose of any matrix $A$ to be the conjugate of $A^t$.
\item No. Let $x=(1,0)$, $y=(0,1)$, and $z=(0,-1)$. Then we have $\lag x,y\rag =\lag x,z\rag =0$.
\item Yes. This means $\|y\|=\lag y,y\rag =0$.
\end{enumerate}
\item Compute 
\[\lag x,y\rag =2\cdot (2+i)+(1+i)\cdot 2+i\cdot (1-2i)=8+5i,\] 
\[\|x\|=\lag x,x\rag ^{\frac{1}{2}}=(4+2+1)^{\frac{1}{2}}=\sqrt{7},\]
\[\|y\|=\lag y,y\rag ^{\frac{1}{2}}=(5+4+5)^{\frac{1}{2}}=\sqrt{14},\]
and 
\[\|x+y\|=\|(4-i,3+i,1+3i)\|=(17+10+10)^{\frac{1}{2}}=\sqrt{37}.\]
We have the Cauchy-Schwarz inequality and the triangle inequality hold since
\[|8+5i|=\sqrt{89}\leq \sqrt{7}\sqrt{14}=\sqrt{98}\]
and 
\[\sqrt{7}+\sqrt{14}\geq \sqrt{37}.\]
\item By definition, we have
\[\begin{aligned}
    \lag f,g\rag &= \int_0^1{f(t)g(t)dt}=\int{tde^t}=te^t|_0^1-\int{e^tdt}=e-(e-1)=1, \\
    \|f\|^2 &= \int_0^1{f^2}=\frac{1}{3} \text{ and } \|f\| = \frac{1}{\sqrt{3}}, \\
    \|g\|^2 &= \int_0^1{g^2}=\frac{1}{2}(e^2-1) \text{ and } \|g\| = \sqrt{\frac{1}{2}(e^2-1)}, \\
    \|f+g\| &= (\|f\|^2+2\lag f,g\rag +\|g\|^2)^{\frac{1}{2}} = \sqrt{\frac{1}{3} + 2 + \frac{1}{2}(e^2-1)} = \sqrt{\frac{1}{2}e^2 + \frac{11}{6}}.
\end{aligned}\]
And the two inequalities hold since
\[1^2 \leq \frac{1}{3}\cdot \frac{1}{2}(e^2-1)\]
and 
\[\sqrt{\frac{1}{2}e^2 + \frac{11}{6}} \geq \frac{1}{\sqrt{3}}+ \sqrt{\frac{1}{2}(e^2-1)}.\]
\item \begin{enumerate}
\item We prove the formula 
\[\lag A,B\rag =\tr(B^*A)=\sum_{j=1}^n{(B^*A)_{jj}}=\sum_{j=1}^n{\sum_{i=1}^n{B^*_{ji}A_{ij}}}\]
\[=\sum_{j=1}^n{\sum_{i=1}^n{A_{ij}\overline{B}_{ij}}}=\sum_{i=1}^n{\sum_{j=1}^n{A_{ij}\overline{B}_{ij}}}=\sum_{i,j}{A_{ij}\overline{B}_{ij}}.\]
So we may view the space $M_{n\times n}(\F)$ to be $F^{n^2}$ and the Frobenius inner product is corresponding to the standard inner product in $F^{n^2}$.
\item Also use the formule to compute
\[\|A\|=(1+5+9+1)^{\frac{1}{2}}=4,\]
\[\|B\|=(2+0+1+1)^{\frac{1}{2}}=2,\]
and 
\[\lag A,B\rag =(1-i)+0-3i-1=-4i.\]
\end{enumerate}
\item We prove the condition for an inner product space one by one.
\begin{itemize}
\item \[\lag x+z,y\rag =(x+z)Ay^*=xAY^*+zAy^*=\lag x,y\rag +\lag z,y\rag .\]
\item \[\lag cx,y\rag =(cx)Ay^*=c(xAy^*)=c\lag x,y\rag .\]
\item \[\overline{\lag x,y\rag }=(xAy^*)^*=yA^*x^*=yAx^*=\lag y,x\rag .\]
One of the equalities uses the fact $A=A^*$.
\item \[\lag x,x\rag =(x_1,x_2)A(x_1,x_2)^* = |x_1|^2 + ix_1\overline{x_2}-ix_2\overline{x_1}+2|x_2|^2\]
\[= |x_1|^2 + 2\Re(ix_1\overline{x_2})+2|x_2|^2 = |x_1 - ix_2|^2 + |x_2|^2 > 0\]
if $x_1$ or $x_2$ is not $0$. Here the function $\Re(z)$ means the real part of a complex number $z$.
\end{itemize}
So it's an inner product function. Also compute that 
\[\lag x,y\rag =1(1-i)(2-i)+i(1-i)(3+2i)+(-i)(2+3i)(2-i)+2(2+3i)(3+2i)\]
\[=(1-i)(2i)+(2+3i)(5+2i)=6+21i.\]
\item \begin{enumerate}
\item \[\lag x,y+z\rag =\overline{\lag y+z,x\rag }=\overline{\lag y,x\rag }+\overline{\lag z,x\rag }=\lag x,y\rag +\lag x,z\rag .\]
\item \[\lag x,cy\rag =\overline{\lag cy,x\rag }=\overline{c\lag y,x\rag }=\overline{c}\lag x,y\rag .\]
\item \[\lag x,0\rag =\overline{0}\lag x,0\rag =0\]
and 
\[\lag 0,x\rag =\overline{\lag x,0\rag }=0.\]
\item If $x=0$, then $\lag 0,0\rag =0$ by previous rule. If $x\neq 0$, then $\lag x,x\rag >0$.
\item If $\lag x,y\rag =\lag x,z\rag $ for all $x\in V$, we have $\lag x,y-z\rag =0$ for all $x\in V$. So we have $\lag y-z,y-z\rag =0$ and hence $y-z=0$.
\end{enumerate}
\item \begin{enumerate}
\item \[\|cx\|=\lag cx,cx\rag ^{\frac{1}{2}}=(c\overline{c}\lag x,x\rag)^{\frac{1}{2}}=|c|\|x\|.\]
\item This is the result of $\|x\|=\lag x,x\rag^{\frac{1}{2}}$ and Theorem 6.1(d).
\end{enumerate}
\item \begin{enumerate}
\item The inner product of a nonzero vector $(1,1)$ and itself is $1-1=0$.
\item Let $A=B=I_2$. We have 
\[\lag 2A,B\rag=3\neq 2\lag A,B\rag =4.\]
\item The inner product of a nonzero function $f(x)=1$ and itself is 
\[\int_0^1{0\cdot 1dx}=0.\]
\end{enumerate}
\item \begin{enumerate}
\item Represent $x$ to be a linear combination of $\beta$ as 
\[x=\sum_{i=1}^k{a_iz_i},\]
where $z_i$'s are elements in $\beta$. Then we have 
\[\lag x,x\rag =\lag x,\sum_{i=1}^k{a_iz_i}\rag \]
\[=\sum_{i=1}^k{\overline{a_i}\lag x,z_i\rag}=0\]
and so $x=0$.
\item This means that $\lag x-y,z\rag =0$ for all $z\in \beta$. So we have $x-y=0$ and $x=y$.
\end{enumerate}
\item Two vectors are orthogonal means that the inner product of them is $0$. So we have 
\[\|x+y\|^2=\lag x+y,x+y\rag \]
\[=\|x\|^2+\lag x,y\rag +\lag y,x\rag +\|y\|^2=\|x\|^2+ +\|y\|^2.\]
To deduce the Pythagorean Theorem in $R^2$, we may begin by a right triangle $ABC$ with the right angle $B$. Assume that $x=AB$ and $y=BC$. Thus we know the length of two leg is $\|x\|$ and $\|y\|$. Finally we know $AC=x+y$ and so the length of the hypotenuse is $\|x+y\|$. Apply the proven equality and get the desired result.
\item Compute that 
\[\|x+y\|^2+\|x-y\|^2\]
\[=(\|x\|^2+\lag x,y\rag +\lag y,x\rag +\|y\|^2)+(\|x\|^2-\lag x,y\rag -\lag y,x\rag +\|y\|^2)\]
\[=2\|x\|^2+2\|y\|^2.\]
This means that the sum of square of the four edges of a parallelogram is the sum of square of the two diagonals.
\item Compute that 
\[\|\sum_{i=1}^k{a_iv_i}\|^2=\lag \sum_{i=1}^k{a_iv_i}, \sum_{i=1}^k{a_iv_i} \rag \]
\[=\sum_{i=1}^k{\|a_iv_i\|^2}+\sum_{i,j}\lag a_iv_i,a_jv_j\rag \]
\[=\sum_{i=1}^k{|a_i|^2\|v_i\|}.\]
\item Check those condition.
\begin{itemize}
\item \[\lag x+z,y\rag =\lag x+z,y\rag_1+\lag x+z,y\rag_2\]
\[=\lag x,y\rag_1+\lag z,y\rag_1+\lag x,y\rag_2+\lag z,y\rag_2\]
\[=\lag x,y\rag +\lag z,y\rag .\]
\item \[\lag cx,y\rag =\lag cx,y\rag_1+\lag cx,y\rag_2\]
\[=c\lag x,y\rag_1+c\lag x,y\rag_2=c\lag x,y\rag .\]
\item \[\overline{\lag x,y\rag}=\overline{\lag x,y\rag_1}+\overline{\lag x,y\rag_2}\]
\[=\lag y,x\rag_1+\lag y,x\rag_2=\lag y,x\rag .\]
\item \[\lag x,x\rag =\lag x,x\rag_1+\lag x,x\rag_2>0\]
if $x\neq 0$.
\end{itemize}
\item Check that 
\[(A+cB)^*_{ij}=\overline{(A+cB)_{ji}}=\overline{A_{ji}}+\overline{c}\overline{B_{ji}}=(A^*+\overline{c}B^*)_{ij}.\]
\item \begin{enumerate}
\item If one of $x$ or $y$ is zero, then the equality holds naturally and we have $y=0x$ or $x=0y$. So we may assume that $y$ is not zero. Now if $x=cy$, we have 
\[|\lag x,y\rag |=|\lag cy,y\rag|=|c|\|y\|^2\]
and 
\[\|x\|\cdot \|y\|=\|cy\|\cdot \|y\|=|c|\|y\|^2.\]
For the necessity, we just observe the proof of Theorem 6.2(c). If the equality holds, then we have 
\[\|x-cy\|=0,\]
where 
\[c=\frac{\lag x,y\rag}{\lag y,y\rag}.\]
And so $x=cy$.
\item Recall that one may derive the Cauchy--Schwarz inequality from the triangle inequality, and vise versa.  This is due to the connection that 
\[
    \|x + y\|^2 = \|x\|^2 + 2\lag x, y\rag + \|y\|^2
\]
and  
\[
    (\|x\| + \|y\|)^2 = \|x\|^2 + 2\|x\|\|y\| + \|y\|^2.
\]
Therefore, $|\lag x,y\rag| = \|x\|\|y\|$ if and only if $\|x + y\| = \|x\| + \|y\|$.  

Alternatively, one may also observe the proof of Theorem 6.2(d). The equality holds only when 
\[\Re\lag x,y\rag =|\lag x,y\rag |=\|x\|\cdot \|y\|.\]
The case $y=0$ is easy. Assuming $y\neq 0$, we have $x=cy$ for some scalar $c\in F$. And thus we have
\[\Re(c)\|y\|^2=\Re\lag cy,y\rag =|\lag cy,y\rag |=|c|\cdot \|y\|^2\]
and so 
\[\Re(c)=|c|.\]
This means $c$ is a nonnegative real number. Conversely, if $x=cy$ for some nonnegative real number, we may check 
\[\|x+y\|=|c+1|\|y\|=(c+1)\|y\|\]
and 
\[\|x\|+\|y\|=|c|\|y\|+\|y\|=(c+1)\|y\|.\]

Finally, we may generalize it to the case of $n$ vectors. That is, 
\[\|x_1+x_2+\cdots +x_n\|=\|x_1\|+\|x_2\|+\cdots +\|x_n\|\]
if and only if we may pick one vector from them and all other vectors are some multiple of that vector using nonnegative real number.
\end{enumerate}
\item \begin{enumerate}
\item Check the condition one by one.\begin{itemize}
\item \[\lag f+h,g\rag=\frac{1}{2\pi}\int_0^{2\pi}{(f(t)+h(t))\overline{g(t)}dt}\]
\[=\frac{1}{2\pi}\int_0^{2\pi}{f\overline{g}dt}+\frac{1}{2\pi}\int_0^{2\pi}{h\overline{g}dt}=\lag f,g\rag +\lag h,g\rag.\]
\item \[\lag cf,g\rag=\frac{1}{2\pi}\int_0^{2\pi}{cf\overline{g}dt}\]
\[=c(\frac{1}{2\pi}\int_0^{2\pi}{f\overline{g}dt})=c\lag f,g\rag .\]
\item \[\overline{\lag f,g\rag}=\overline{\frac{1}{2\pi}\int_0^{2\pi}{f\overline{g}dt}}=\frac{1}{2\pi}\int_0^{2\pi}{\overline{f\overline{g}}dt}\]
\[=\frac{1}{2\pi}\int_0^{2\pi}{g\overline{f}dt}=\lag g,f\rag.\]
\item \[\lag f,f\rag =\frac{1}{2\pi}\int_0^{2\pi}{\|f\|^2dt}>0\]
if $f$ is not zero. Ur... I think this is an exercise for the Adavanced Calculus course.
\end{itemize}
\item No. Let 
\[f(t)=\left\{\begin{array}{ccc}0&\mathrm{if} &x\leq \frac{1}{2};\\x-\frac{1}{2}&\mathrm{if} &x>\frac{1}{2}.\end{array}\right..\]
Then we have that $\lag f,f\rag=0$ but $f\neq 0$.
\end{enumerate}
\item If $T(x)=0$ Then we have 
\[\|x\|=\|0\|=0.\]
This means $x=0$ and so $T$ is injective.
\item If $\lag \cdot, \cdot \rag'$ is an inner product on $V$, then we have that $T(x)=0$ implies 
\[\lag x,x\rag'=\lag T(x),T(x)\rag =0\]
and $x=0$. So $T$ is injective. Conversely, if $T$ is injective, we may check those condition for inner product one by one.
\begin{itemize}
\item \[\lag x+z,y\rag'=\lag T(x+z),T(y)\rag =\lag T(x)+T(z),T(y)\rag\]
\[=\lag T(x),T(y)\rag +\lag T(z),T(y)\rag =\lag x,y\rag' +\lag z,y\rag' .\]
\item \[\lag cx,y\rag'=\lag T(cx),T(y)\rag =\lag cT(x),T(y)\rag\]
\[=c\lag T(x),T(y)\rag =\lag x,y\rag'.\]
\item \[\overline{\lag x,y\rag'}=\overline{\lag T(x),T(y)\rag}\]
\[=\lag T(y),T(x)\rag =\lag y,x\rag'.\]
\item \[\lag x,x\rag'=\lag T(x),T(x)\rag>0\]
if $T(x)\neq 0$. And the condition $T(x)\neq 0$ is true when $x\neq 0$ since $T$ is injective.
\end{itemize}
\item \begin{enumerate}
\item Just compute 
\[\|x+y\|^2=\|x\|^2+\lag x,y\rag +\lag y,x\rag +\|y\|^2\]
\[=\|x\|^2+\lag x,y\rag +\overline{\lag y,x\rag }+\|y\|^2=\|x\|^2+2\Re(\lag x,y\rag )+\|y\|^2\]
and 
\[\|x-y\|^2=\|x\|^2-\lag x,y\rag -\lag y,x\rag +\|y\|^2\]
\[=\|x\|^2-\lag x,y\rag -\overline{\lag y,x\rag }+\|y\|^2=\|x\|^2-2\Re(\lag x,y\rag )+\|y\|^2.\]
\item We have 
\[\|x-y\|+\|y\|\geq \|x\|\]
and 
\[\|x-y\|+\|x\|=\|y-x\|+\|x\|\geq \|y\|.\]
Combining these two we get the desired inequality.
\end{enumerate}
\item \begin{enumerate}
\item By Exercise 6.1.19(a), we have the right hand side would be 
\[\frac{1}{4}(4\Re\lag x,y\rag)=\lag x,y\rag.\]
The last equality is due to the fact $\F=\R$.
\item Also use the Exercise 6.1.19(a). First observe that 
\[\|x+i^ky\|^2=\|x\|^2+2\Re\lag x,i^ky\rag +\|i^ky\|^2\]
\[=\|x\|^2+2\Re[\overline{i^k}\lag x,y\rag ]+\|y\|^2.\]
Assuming $\lag x,y\rag$ to be $a+bi$, we have the right hand side would be
\[\frac{1}{4}[(\sum_{k=1}^4{i^k})\|x\|^2+2\sum_{k=1}^4{i^k\Re[\overline{i^k}\lag x,y\rag ]}+(\sum_{k=1}^4{i^k})\|y\|^2]\]
\[=\frac{1}{2}[bi+(-a)(-1)+(-b)(-i)+a]=a+bi=\lag x,y\rag.\]
\end{enumerate}
\item \begin{enumerate}
\item Observe that 
\[A_1^*=(\frac{1}{2}(A+A^*))^*=\frac{1}{2}(A^*+A)=A_1^*,\]
\[A_2^*=(\frac{1}{2i}(A-A^*))^*=(-\frac{1}{2i}(A^*-A)=A_2^*.\]
and 
\[A_1+iA_2=\frac{1}{2}(A+A^*)+\frac{1}{2}(A-A^*)=\frac{1}{2}(A+A)=A.\]
I don't think it's reasonable because $A_1$ does not consists of the real part of all entries of $A$ and $A_2$ does not consists of the imaginary part of all entries of $A$. But what's the answer do you want? Would it be reasonable to ask such a strange question?
\item If we have $A=B_1+iB_2$ with $B_1^*=B_1$ and $B_2^*=B_2$, then we have $A^*=B_1^*-iB_2^*$. Thus we have 
\[B_1=\frac{1}{2}(A+A^*)\]
and 
\[B_2=\frac{1}{2i}(A-A^*).\]
\end{enumerate}
\item \begin{enumerate}
\item As definition, we may find $v_1,v_2,\ldots ,v_n\in \beta$ for $x,y,z\in V$ such that 
\[x=\sum_{i=1}^n{a_iv_i},y=\sum_{i=1}^n{b_iv_i},z=\sum_{i=1}^n{d_iv_i}.\]
And check those condition one by one.
\begin{itemize}
\item \[\lag x+z,y\rag=\lag \sum_{i=1}^n{(a_i+d_i)v_i},\sum_{i=1}^n{b_iv_i}\rag \]
\[=\sum_{i=1}^n{(a_i+d_i)\overline{b_i}}=\sum_{i=1}^n{a_i\overline{b_i}}+\sum_{i=1}^n{d_i\overline{b_i}}\]
\[=\lag x,y\rag+\lag z,y\rag.\]
\item \[\lag cx,y\rag=\lag \sum_{i=1}^n{(ca_i)v_i},\sum_{i=1}^n{b_iv_i}\rag \]
\[=\sum_{i=1}^n{(ca_i)\overline{b_i}}=c\sum_{i=1}^n{a_i\overline{b_i}}=c\lag x,y\rag .\]
\item \[\overline{\lag x,y\rag}=\overline{\sum_{i=1}^n{a_i\overline{b_i}}}\]
\[=\sum_{i=1}^n{\overline{a_i\overline{b_i}}}=\sum_{i=1}^n{b_i\overline{a_i}}=\lag y,x\rag .\]
\item \[\lag x,x\rag =\sum_{i=1}^n{|a_i|^2}>0\]
if all $a_i$'s is not zero. That is, $x$ is not zero.
\end{itemize}
\item If the described condition holds, for each vector 
\[x=\sum_{i=1}^n{a_iv_i}\] 
we have actually $a_i$ is the $i$-th entry of $x$. So the function is actually the atandard inner product. Note that this exercise give us an idea that different basis will give a different inner product.
\end{enumerate}
\item \begin{enumerate}
\item We have the fact that with the standard inner product $\lag \cdot, \cdot \rag$ we have $\lag x,y\rag=y^*x$. So we have 
\[\lag x,Ay\rag =(Ay)^*x=y^*A^*x=\lag A^*x,y\rag.\]
\item First we have that 
\[\lag A^*x,y\rag =\lag x,Ay\rag =\lag Bx,y\rag \]
for all $x,y\in V$. By Theorem 6.1(e) we have $A^*x=Bx$ for all $x$. But this means that these two matrix is the same.
\item Let $\beta=\{v_1,v_2,\ldots ,v_n\}$. So the column vectors of $Q$ are those $v_i$'s. Finally observe that $(Q^*Q)_{ij}$ is 
\[v_i^*v_j=\lag v_i,v_j\rag =\left\{\begin{array}{ccc}1&\mathrm{if} &i=j;\\0&\mathrm{if} &i\neq j.\end{array}\right.\]
So we have $Q^*Q=I$ and $Q^*=Q^{-1}$.
\item Let $\alpha $ be the standard basis for $\F^n$. Thus we have $[T]_{\alpha}=A$ and $[U]_{\alpha}=A^*$. Also we have that actually $[I]_{\alpha}^{\beta}$ is the matrix $Q$ defined in the previous exercise. So we know that 
\[[U]_{\beta}=[I]_{\alpha}^{\beta}[U]_{\alpha}[I]_{\beta}^{\alpha}=QA^*Q^{-1}=QA^*Q^*\]
\[=(QAQ^*)^* = ([I]_{\alpha}^{\beta}[T]_{\alpha}[I]_{\beta}^{\alpha})^* = ([T]_{\beta})^*.\]
\end{enumerate}
\item Check the three conditions one by one.\begin{enumerate}
\item \begin{itemize}
\item \[\|A\|=\max_{i,j}|A_{ij}|\geq 0,\]
and the value equals to zero if and only if all entries of $A$ are zero.
\item \[\|aA\|=\max_{i,j}|(aA)_{ij}|=\max_{i,j}|a||A_{ij}|\]
\[=|a|\max_{i,j}|A_{ij}|=|a|\|A\|.\]
\item \[\|A+B\|=\max_{i,j}|(A+B)_{ij}|=\max_{i,j}|A_{ij}+B_{ij}|\]
\[\leq \max_{i,j}|A_{ij}|+\max_{i,j}|B_{ij}|=\|A\|+\|B\|.\]
\end{itemize}
\item \begin{itemize}
\item \[\|f\|=\max_{t\in [0,1]}|f(t)|\geq 0,\]
and the value equals to zero if and only if all value of $f$ in $[0,1]$ is zero.
\item \[\|af\|=\max_{t\in [0,1]}|(af)(t)|=\max_{t\in [0,1]}|a||f(t)|\]
\[=|a|\max_{t\in [0,1]}|f(t)|=|a|\|f\|.\]
\item \[\|f+g\|=\max_{t\in [0,1]}|(f+g)(t)|=\max_{t\in [0,1]}|f(t)+g(t)|\]
\[\leq \max_{t\in [0,1]}|f(t)|+\max_{t\in [0,1]}|g(t)|=\|f\|+\|g\|.\]
\end{itemize}
\item \begin{itemize}
\item \[\|f\|=\int_0^1{|f(t)|dt}\geq 0,\]
and the value equals to zero if and only if $f=0$. This fact depend on the continuity and it would be an exercise in the Advanced Calculus coures.
\item \[\|af\|=\int_0^1{|af(t)|dt}=\int_0^1{|a||f(t)|dt}\]
\[=|a|\int_0^1{|f(t)|}=|a|\|f\|.\]
\item \[\|f+g\|=\int_0^1{|f(t)+g(t)|dt}\leq\int_0^1{|f(t)|+|g(t)|dt}\]
\[=\int_0^1{|f(t)|dt}+\int_0^1{|g(t)|dt}=\|f\|+\|g\|.\]
\end{itemize}
\item \begin{itemize}
\item \[\|(a,b)\|=\max\{|a|,|b|\}\geq 0,\]
and the value equals to zero if and only if both $a$ and $b$ are zero.
\item \[\|c(a,b)\|=\max\{|ca|,|cb|\}=\max\{|c||a|,|c||b|\}\]
\[=|c|\max\{|a|,|b|\}=|c|\|(a,b)\|.\]
\item \[\|(a,b)+(c,d)\|=\max\{|a+c|,|b+d|\}\leq \max\{|a|+|c|,|b|+|d|\}\]
\[\leq \max\{|a|,|b|\}+\max\{|c|,|d|\}=\|(a,b)\|+\|(c,d)\|.\]
\end{itemize}
\end{enumerate}
\item By Exercise 6.1.20 we know that if there is an inner product such that $\|x\|^2=\lag x,x\rag $ for all $x\in \R^2$, then we have 
\[\lag x,y\rag =\frac{1}{4}\|x+y\|^2-\frac{1}{4}\|x-y\|.\]
Let $x=(2,0)$ and $y=(1,3)$. Thus we have 
\[\lag x,y\rag =\frac{1}{4}\|(3,3)\|^2-\frac{1}{4}\|(1,-3)\|^2=0\]
and 
\[\lag 2x,y\rag =\frac{1}{4}\|(5,3)\|^2-\frac{1}{4}\|(3,-3)\|^2=\frac{1}{4}(25-9)=4.\]
This means this function is not linear in the first component.
\item \begin{enumerate}
\item \[d(x,y)=\|x-y\|\geq 0.\]
\item \[d(x,y)=\|x-y\|=\|y-x\|=d(y,x).\]
\item \[d(x,y)=\|x-y\|=\|(x-z)+(z-y)\|\leq \|x-z\|+\|z-y\|=d(x,y)+d(z,y).\]
\item \[d(x,x)=\|x-x\|=\|0\|=0.\]
\item \[d(x,y)=\|x-y\|>0\]
if $x-y$ is not zero. That is, the distance is not zero if $x\neq y$.
\end{enumerate}
\item The third and the fourth condition of inner product naturally hold since 
\[\lag x,y\rag =\frac{1}{4}[\|x+y\|^2-\|x-y\|^2]=\lag y,x\rag\]
and 
\[\lag x,x\rag =\frac{1}{4}[\|2x\|^2-\|0\|^2]=\|x\|^2>0\]
if $x\neq 0$. Now we prove the first two condition as Hints.
\begin{enumerate}
\item Consider that 
\[\|x+2y\|^2+\|x\|^2=2\|x+y\|^2+2\|y\|^2\]
and 
\[\|x-2y\|^2+\|x\|^2=2\|x-y\|^2+2\|y\|^2\]
by the parallelogram law. By Substracting these two equalities we have 
\[\|x+2y\|^2-\|x-2y\|^2=2\|x+y\|^2-2\|x-y\|^2.\]
And so we have 
\[\lag x,2y\rag =\frac{1}{4}[\|x+2y\|^2-\|x-2y\|^2]\]
\[=\frac{1}{4}[2\|x+y\|^2-2\|x-y\|^2]=2\lag x,y\rag .\]
\item By the previous argument, we direct prove that 
\[\lag x+u,2y\rag =2\lag x,y\rag +2\lag u,y\rag .\]
Similarly begin by 
\[\|x+u+2y\|^2+\|x-u\|^2=2\|x+y\|^2+2\|u+y\|^2\]
and 
\[\|x+u-2y\|^2+\|x-u\|^2=2\|x-y\|^2+2\|u-y\|^2\]
by the parallelograom law. By substracting these two equalities we have 
\[\|x+u+2y\|^2-\|x+u-2y\|^2=2\|x+y\|^2-2\|x-y\|^2+2\|u+y\|^2-2\|u-y\|^2.\]
And so we have 
\[\lag x+u,2y\rag =\frac{1}{4}[\|x+u+2y\|^2-\|x+u-2y\|^2]\]
\[=\frac{1}{4}[2\|x+y\|^2-2\|x-y\|^2+2\|u+y\|^2-2\|u-y\|^2]\]
\[=2\lag x,y\rag +2\lag u,y\rag .\]
\item Since $n$ is a positive integer, we have 
\[\lag nx,y\rag =\lag (n-1)x,y\rag +\lag x,y\rag \]
\[=\lag (n-2)x,y\rag +2\lag x,y\rag =\cdots =n\lag x,y\rag \]
by the previous argument inductively.
\item Since $m$ is a positive integer, we have 
\[\lag x,y\rag =\lag m(\frac{1}{m}x),y\rag =m\lag \frac{1}{m}x,y\rag \]
by the previous argument.
\item Let $r=\frac{p}{q}$ for some positive integers $p$ and $q$ if $r$ is positive. In this case we have 
\[\lag rx,y\rag =\lag p(\frac{1}{q}x),y\rag =p\lag \frac{1}{q}x,y\rag \]
\[=\frac{p}{q}\lag x,y\rag =r\lag x,y\rag .\]
If $r$ is zero, then it's natural that 
\[\lag 0,y\rag =\frac{1}{4}[\|y\|^2-\|y\|^2]=0=0\lag x,y\rag .\]
Now if $r$ is negative, then we also have 
\[\lag rx,y\rag =\lag (-r)(-x),y\rag =-r\lag -x,y\rag .\]
But we also have 
\[\lag -x,y\rag =-\lag x,y\rag \]
since 
\[\lag -x,y\rag +\lag x,y\rag \]
\[=\frac{1}{4}[\|-x+y\|^2-\|-x-y\|^2+\|x+y\|^2-\|x-y\|^2]=0.\]
So we know that even when $r$ is negative we have 
\[\lag rx,y\rag =-r\lag -x,y\rag =r\lag x,y\rag .\]
\item By the triangle inequality of the norm, we have 
\[\begin{aligned}
    \|x + y\| &\leq \|x\| + \|y\| \text{ and }\\
    \|x - y\| &\geq \|x\| - \|y\|
\end{aligned}\]
since $\|x - y\| + \|y\| \geq \|x\|$.  Therefore, 
\[\begin{aligned}
    \lag x,y \rag &= \frac{1}{4}\left[\|x + y\|^2 - \|x - y\|^2\right] \\
     &\leq \frac{1}{4}(\|x\| + \|y\|)^2 - \frac{1}{4}(\|x\| - \|y\|)^2 \\
     &= \|x\|\|y\|.
\end{aligned}\]
Replacing $y$ by $-y$ in the inequality, we get $-\lag x,y\rag \leq \|x\|\|y\|$.  Combining these two inequalities, we have the desired result.
\item Since 
\[(c-r)\lag x,y\rag =c\lag x,y\rag -r\lag x,y\rag\]
and 
\[\lag (c-r)x,y\rag =\lag cx-rx,y\rag =\lag cx,y\rag -r\lag x,y\rag ,\]
the first equality holds. And by the previous argument we have 
\[-|c-r|\|x\|\|y\|\leq (c-r)\lag x,y\rag ,\lag (c-r)x,y\rag \leq |c-r|\|x\|\|y\|\]
and so we get the final inequality.
\item For every real number $c$, we could find a rational number such that $|c-r|$ is small enough\footnote{This is also an exercise for the Adavanced Calculus course.}. So by the previous argument, we have 
\[\lag cx,y\rag =c\lag x,y\rag \]
for all real number $c$.
\end{enumerate}
\item Check the conditions one by one. \begin{itemize}
\item \[[x+z,y]=\Re\lag x+z,y\rag =\Re(\lag x,y\rag +\lag z,y\rag)\]
\[=\Re\lag x,y\rag +\Re\lag z,y\rag =[x,y]+[z,y].\]
\item \[[cx,y]=\Re\lag cx,y\rag \]
\[=c\Re\lag x,y\rag =c[x,y],\]
where $c$ is a real number.
\item \[[x,y]=\Re\lag x,y\rag =\Re\overline{\lag x,y\rag }\]
\[=\Re\lag y,x\rag =[y,x].\]
\item \[[x,x]=\Re\lag x,x\rag =\lag x,x\rag >0\]
if $x\neq 0$.
\end{itemize}
Finally, we have $[x,ix]=0$ since 
\[\lag x,ix\rag =-i\lag x,x\rag \]
is a pure imaginary number.
\item Observe that 
\[0=[x+iy,i(x+iy)]=[x+iy,ix-y]=[ix,iy]-[x,y].\]
So we have an instant property 
\[[x,y]=[ix,iy].\]
Now check the conditions one by one. \begin{itemize}
\item \[\lag x+z,y\rag =[x+z,y]+i[x+z,iy]\]
\[=[x,y]+[z,y]+i[x,iy]+i[z,iy]=\lag x,y\rag +\lag z,y\rag .\]
\item \[\lag (a+bi)x,y\rag =[(a+bi)x,y]+i[(a+bi)x,iy]\]
\[=[ax,y]+[bix,y]+i[ax,iy]+i[bix,iy]\]
\[=a([x,y]+i[x,iy])+bi([ix,iy]-i[ix,y])\]
\[=a\lag x,y\rag +bi([x,y]+i[x,iy])=(a+bi)\lag x,y\rag .\]
Here we use the proven property.
\item \[\overline{\lag x,y\rag}=[x,y]-i[x,iy]\]
\[=[y,x]+i[y,ix]=\lag x,y\rag .\]
Here we use the proven property again.
\item \[\lag x,x\rag =[x,x]+i[x,ix]=[x,x]>0\]
if $x$ is not zero.
\end{itemize}
\item First we may observe that the condition for norm on real vector space is loosen than that on complex vector space. So naturally the function $\|\cdot \|$ is still a norm when we regard $V$ as a vector space over $\R$. By Exercise 6.1.27, we've already defined a real inner product $[\cdot ,\cdot ]$ on it since the parallelogram law also holds on it. And we also have 
\[[x,ix]=\frac{1}{4}[\|x+ix\|^2-\|x-ix\|^2]=\]
\[=\frac{1}{4}[\|x+ix\|^2-\|(-i)(x+ix)\|^2]=\frac{1}{4}[\|x+ix\|^2-|-i|\|(x+ix)\|^2]=0.\]
So by Exercise 6.1.29 we get the desired conclusion.
\end{enumerate}
