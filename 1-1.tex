\section{Introduction}
\begin{enumerate}
\item \begin{enumerate}
\item No. \(\frac{3}{6} \neq \frac{1}{4}\)
\item Yes. $-3(-3,1,7)=(9,-3,-21)$
\item No.
\item No.
\end{enumerate}
\item Here $t$ is in $\mathbb{F}$.

\begin{enumerate}
\item $(3,-2,4)+t(-8,9,-3)$
\item $(2,4,0)+t(-5,-10,0)$
\item $(3,7,2)+t(0,0,-10)$
\item $(-2,-1,5)+t(5,10,2)$
\end{enumerate}
\item Here $s$ and $t$ are in $\mathbb{F}$.

\begin{enumerate}
\item $(2,-5,-1)+s(-2,9,7)+t(-5,12,2)$
\item $(3,-6,7)+s(-5,6,-11)+t(2,-3,-9)$
\item $(-8,2,0)+s(9,1,0)+t(14,3,0)$
\item $(1,1,1)+s(4,4,4)+t(-7,3,1)$
\end{enumerate}
\item Additive identity, $0$, should be the zero vector, $(0,0,\dots ,0)$ in $\mathbb{R}^n $.
\item Since $x=(a_1,a_2)-(0,0)=(a_1,a_2)$, we have $tx=(ta_1,ta_2)$. Hence the head of that vector will be $(0,0)+(ta_1,ta_2)=(ta_1,ta_2)$.
\item The vector that emanates from $(a,b)$ and terminates at the midpoint should be $\frac{1}{2}(c-a,d-b)$. So the coordinate of the midpoint will be $(a,b)+\frac{1}{2}(c-a,d-b)=((a+c)/2,(b+d)/2)$.
\item Let the four vertices of the parallelogram be $A$, $B$, $C$, $D$ counterclockwise. Say $x=\vec{AB}$ and $y=\vec{AD}$. Then the line joining points $B$ and $D$ should be $x+s(y-x)$, where $s$ is in $\mathbb{F}$. And the line joining points $A$ and $C$ should be $t(x+y)$, where $t$ is in $\mathbb{F}$. To find the intersection of the two lines we should solve $s$ and $t$ such that $x+s(y-x)=t(x+y)$. Hence we have $(1-s-t)x=(t-s)y$. But since $x$ and $y$ can not be parallel, we have $1-s-t=0$ and $t-s=0$. So $s=t=\frac{1}{2}$ and the midpoint would be the head of the vector $\frac{1}{2}(x+y)$ emanating from $A$ and by the previous exercise we know it's the midpoint of segment $AC$ or segment $BD$.
\end{enumerate}