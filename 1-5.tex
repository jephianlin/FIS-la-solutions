\section{Linear Dependence and Linear Independence}
\begin{enumerate}
\item \begin{enumerate}
\item No. For example, take $S=\{(1,0),(2,0),(0,1)\}$ and then $(0,1)$ is not a linear combination of the other two.
\item Yes. It's because $1\vec{0}=\vec{0}$.
\item No. It's independent by the remark after Definition of linearly independent.
\item No. For example, we have $S=\{(1,0),(2,0),(0,1)\}$ but $\{(1,0),(0,1)\}$ is linearly independent.
\item Yes. This is the contrapositive statement of Theorem 1.6.
\item Yes. This is the definition.
\end{enumerate}
\item \begin{enumerate}
\item Linearly dependent. We have $-2\left( \begin{array}{cc}1&-3\\-2&4\end{array}\right) =\left( \begin{array}{cc}-2&6\\4&-8\end{array}\right)$. So to check the linearly dependency is to find the nontrivial solution of equation $a_1x_1+a_2x_2+\cdots +a_nx_n=0$. And $x_1$ and $x_2$ are the two matrices here.
\item Linearly independent.
\item Linearly independent.
\item Linearly dependent.
\item Linearly dependent.
\item Linearly independent.
\item Linearly dependent.
\item Linearly independent.
\item Linearly independent.
\item Linearly dependent.
\end{enumerate}
\item Let $M_1, M_2,\dots ,M_5$ be those matrices. We have $M_1+M_2+M_3-M_4-M_5=0$.
\item If $a_1e_1+a_2e_2+\cdots +a_ne_n=(a_1,a_2,\dots ,a_n)=0$, then by comparing the $i$-th entry of the vector of both side we have $a_1=a_2=\cdots =a_n=0$.
\item It's similar to exercise 1.5.4.
\item it's similar to exercise 1.5.4.
\item Let $E_{ij}$ be the matrix with the only nonzero $ij$-entry$=1$. Then $\{E_{11},E_{22}\}$ is the generating set.
\item \begin{enumerate}
\item The equation $x_1(1,1,0)+x_2(1,0,1)+x_3(0,1,1)=0$ has only nontrivial solution when $\mathbb{F}=\mathbb{R}$.
\item When $\mathbb{F}$ has characteristic 2, we have $1+1=0$ and so $(1,1,0)+(1,0,1)+(0,1,1)=(0,0,0)$.
\end{enumerate}
\item It's sufficient since if $u=tv$ for some $t\in \mathbb{F}$ then we have $u-tv=0$. While it's also necessary since if $au+bv=0$ for some $a,b\in \mathbb{F}$ with at least one of the two coefficients not zero then we may assume $a\neq0$ and $u=-\frac{b}{a}v$.
\item Pick $v_1=(1,1,0)$, $v_2=(1,0,0)$, $v_3=(0,1,0)$. And we have that none of the three is a multiple of another and they are dependent since $v_1-v_2-v_3=0$.
\item Vector in span$(S)$ are linear combinations of $S$ and they all have different representation by the remark after Definition of linear independent. So there are $2^n$ representations and so $2^n$ vectors.
\item Since $S_1$ is linearly dependent we have finite vectors $x_1,x_2,\ldots ,x_n$ in $S_1$ and so in $S_2$ such that $a_1x_1+a_2x_2+\cdots +a_nx_n=0$ is a nontrivial representation. But the nontrivial representation is also a nontrivial representation of $S_2$. And the Corollary is just the contrapositive statement of the Theorem 1.6.
\item \begin{enumerate}
\item Sufficiency: If $\{u+v,u-v\}$ is linearly independent we have $a(u+v)+b(u-v)=0$ implies $a=b=0$. Assuming that $cu+dv=0$, we can deduce that $\frac{c+d}{2}(u+v)+\frac{c-d}{2}(u-v)=0$ and hence $\frac{c+d}{2}=\frac{c-d}{2}=0$. This means $c=d=0$ if the characteristc is not two. Necessity: If $\{u,v\}$ is linearly independent we have $au+bv=0$ implies $a=b=0$. Assuming that $c(u+v)+d(u-v)=0$, we can deduce that $(c+d)u+(c-d)v=0$ and hence $c+d=c-d=0$ and $2c=2d=0$. This means $c=d=0$ if the characteristc is not two.
\item Sufficiency: If $au+bv+cw=0$ we have $\frac{a+b-c}{2}(u+v)+\frac{a-b+c}{2}(u+w)+\frac{-a+b+c}{2}(v+w)=0$ and hence $a=b=c=0$. Necessity: If $a(u+v)+b(u+w)+c(v+w)=0$ we have $(a+b)u+(a+c)v+(b+c)w=0$ and hence $a=b=c=0$.
\end{enumerate}
\item Sufficiency: It's natural that ${0}$ is linearly dependent. If $v$ is a linear combination of $u_1,u_2, \ldots ,u_n$ , say $v=a_1u_1+a_2u_2+\cdots a_nu_n$, then $v-a_1u_1-a_2u_2-\cdots -a_nu_n=0$ implies $S$ is linearly dependent. Necessity: If $S$ is linearly dependent and $S\neq\{0\}$ we have some nontrivial representation $a_0u_0+a_1u_1+\cdots +a_nu_n=0$ with at least one of the coefficients is nonzero, say $a_0\neq 0$ without loss the generality. Then we can let $v=u_0=-\frac{1}{a_0}(a_1u_1+a_2u_2+\cdots +a_nu_n)$.
\item Sufficiency: If $u_1=0$ then $S$ is linearly independent. If \[u_{k+1}\in \mathrm{span}(\{u_1,u_2,\ldots ,u_k\})\] for some $k$, say $u_{k+1}=a_1u_1+a_2u_2+\cdots +a_ku_k$, then we have $a_1u_1+a_2u_2+\cdots +a_ku_k-u_{k+1}=0$ is a nontrivial representation. Necessary: If $S$ is linearly dependent, there are some integer $k$ such that there is some nontrivial representation $a_1u_1+a_2u_2+\cdots +a_ku_k+a_{k+1}u_{k+1}=0$. Furthermore we may assume that $a_{k+1}\neq 0$ otherwise we may choose less $k$ until that $a_{k+1}\neq =0$. Hence we have $a_{k+1}=-\frac{1}{a_{k+1}}(a_1u_1+a_2u_2+\cdots +a_ku_k)$ and so $a_{k+1}\in \mathrm{span}(\{u_1,u_2,\ldots ,u_k\})$.
\item Sufficiency: We can prove it by contrapositive statement. If $S$ is linearly dependent we can find $a_1u_1+a_2u_2+\cdots +a_nu_n=0$. But thus the finite set $\{u_1,u_2,\ldots ,u_n\}$ would be a finite subset of $S$ and it's linearly dependent. Necessary: This is the Threorem 1.6.
\item Let $C_1,C_2, \ldots ,C_n$ be the columns of $M$. Let $a_1C_1+a_2C_2+\cdots +a_nC_n=0$ then we have $a_n=0$ by comparing the $n$-th entry. And inductively we have $a_{n-1}=0,a_{n-2}=0, \ldots , a_{1}=0$.
\item It's similar to exercise 1.5.17.
\item We have $a_1A_1^t+a_2A_2^t+\cdots +a_kA_k^t=0$ implies $a_1A_1+a_2A_2+\cdots +a_kA_k=0$. Then we have $a_1=a_2=\cdots =a_n=0$.
\item If $\{f,g\}$ is linearly dependent, then we have $f=kg$. But this means $1=f(0)=kg(0)=k\times 1$ and hence $k=1$. And $e^r=f(1)=kg(1)=e^s$ means $r=s$.
\end{enumerate}
