\section{Vector Spaces}
\begin{enumerate}
\item
\begin{enumerate}
\item Yes. It's condition (VS 3).
\item No. If $x$, $y$ are both zero vectors. Then by condition (VS 3) $x=x+y=y$.
\item No. Let $e$ be the zero vector. We have $1e=2e$.
\item No. It will be false when $a=0$.
\item Yes.
\item No. It has $m$ rows and $n$ columns.
\item No.
\item No. For example, we have that $x+(-x)=0$.
\item Yes.
\item Yes.
\item Yes. That's the definition.
\end{enumerate}
\item It's the $3\times 4$ matrix with all entries =0.
\item $M_{13}=3$, $M_{21}=4$, $M_{22}=5$.
\item 
\begin{enumerate}
\item \(
\left( \begin{array}{ccc}
6& 3& 2\\
-4& 3& 9
\end{array}\right)
\).
\item \(
\left( \begin{array}{cc}
1& -1\\
3& -5\\
3& 8
\end{array}\right)
\).
\item \(
\left( \begin{array}{ccc}
8& 20& -12\\
4& 0& 28
\end{array}\right)
\).
\item \(
\left( \begin{array}{cc}
30& -20\\
-15& 10\\
-5& -40
\end{array}\right)
\).
\item $2x^4+x^3+2x^2-2x+10$.
\item $-x^3+7x^2+4$.
\item $10x^7-30x^4+40x^2-15x$.
\item $3x^5-6x^3+12x+6$.
\end{enumerate}
\item \(
\left(\begin{array}{ccc}
8& 3& 1\\
3& 0& 0\\
3& 0& 0
\end{array}\right)
+
\left(\begin{array}{ccc}
9& 1& 4\\
3& 0& 0\\
1& 1& 0
\end{array}\right)
=
\left(\begin{array}{ccc}
17& 4& 5\\
6& 0& 0\\
4& 1& 0
\end{array}\right)\).
\item \(M=\left(\begin{array}{cccc}4&2&1&3\\5&1&1&4\\3&1&2&6\end{array}\right)\). Since all the entries has been doubled, we have $2M$ can describe the inventory in June. Next, the matrix $2M-A$ can describe the list of sold items. And the number of total sold items is the sum of all entries of $2M-A$. It equals $24$.
\item It's enough to check $f(0)+g(0)=2=h(0)$ and $f(1)+g(1)=6=h(1)$.
\item By (VS 7) and (VS 8), we have $(a+b)(x+y)=a(x+y)+b(x+y)=ax+ay+bx+by$.
\item For two zero vectors $0_0$ and $0_1$, by Thm 1.1 we have that $0_0+x=x=0_1+x$ implies $0_0=0_1$, where $x$ is an arbitrary vector. If for vector $x$ we have two inverse vectors $y_0$ and $y_1$. Then we have that $x+y_0=0=x+y_1$ implies $y_0=y_1$. Finally we have $0a+1a=(0+1)a=1a=0+1a$ and so $0a=0$.
\item We have sum of two differentiable real-valued functions or product of scalar and one differentiable real-valued function are again that kind of function. And the function $f=0$ would be the $0$ in a vector space. Of course, here the field should be the real numbers.
\item All condition is easy to check because there is only one element.
\item We have $f(-t)+g(-t)=f(t)+g(t)$ and $cf(-t)=cf(t)$ if $f$ and $g$ are both even function. Futhermore, $f=0$ is the zero vector. And the field here should be the real numbers.
\item No. If it's a vector space, we have $0(a_1,a_2)=(0,a_2)$ be the zero vector. But since $a_2$ is arbitrary, this is a contradiction to the uniqueness of zero vector.
\item Yes. All the condition are preserved when the field is the real numbers.
\item No. Because a real-valued vector scalar multiply with a complex number will not always be a real-valued vector.
\item Yes. All the condition are preserved when the field is the rational numbers.
\item No. Since $0(a_1,a_2)=(a_1,0)$ is the zero vector but this will make the zero vector not be unique, it cannot be a vector space.
\item No. We have $((a_1,a_2)+(b_1,b_2))+(c_1,c_2)=(a_1+2b_1+2c_1,a_2+3b_2+3c_2)$ but $(a_1,a_2)+((b_1,b_2)+(c_1,c_2))=(a_1+2b_1+4c_1,a_2+3b_2+9c_2)$.
\item No. Because $(c+d)(a_1,a_2)=((c+d)a_1,\frac{a_2}{c+d})$ may not equal to $c(a_1,a_2)+d(a_1,a_2)=(ca_1+dc_1,\frac{a_2}{c}+\frac{a_2}{d})$.
\item A sequence can just be seen as a vector with countable-infinite dimensions. Or we can just check all the condition carefully.
\item Let $0_V$ and $0_W$ be the zero vector in $V$ and $W$ respectly. Then we have $(0_V,0_W)$ will be the zero vector in $Z$. The other condition could also be checked carefully. This space is called the direct product of $V$ and $W$.
\item Since each entries could be $1$ or $0$ and there are $m\times n$ entries, there are $2^{m\times n}$ vectors in that space.
\end{enumerate}