\section{The Adjoint of a Linear Operator}
\begin{enumerate}
\item \begin{enumerate}
\item Yes. See Theorem 6.9.
\item No. It just for the linear mapping $V\rightarrow \F$. For example, the value of the identity mapping from $\R^2$ to $\R^2$ is not an element in $\F$.
\item No. The equality holds only for the case $\beta $ is an orthonormal basis. For example, let $A=\begin{pmatrix}1&1\\0&1\end{pmatrix}$ and $T=L_A$. Thus we have $T^*=L_{A^*}$. But for the basis 
\[\beta =\{(1,1),(0,1)\},\]
we have 
\[[T]_{\beta}=\begin{pmatrix}2&1\\-1&0\end{pmatrix}\neq [T^*]_{\beta}=\begin{pmatrix}1&0\\1&1\end{pmatrix}.\]
\item Yes.See Theorem 6.9.
\item No. Choose $a=i$, $b=0$ and $T=I_{\mathbb{C}}=U$. We have $(aT)^*=\overline{a}T^*\neq aT^*$.
\item Yes. See Theorem 6.10 and its Corollary.
\item Yes. See Theorem 6.11.
\end{enumerate}
\item Follow the prove of Theorem 6.8. The vector would be 
\[y=\sum_{i=1}^n{\overline{g(v_i)}v_i}.\]
\begin{enumerate}
\item The vector is $(1,-2,4)$.
\item The vector is $(1,-2)$.
\item The vector is $210x^2-204x+33$.
\end{enumerate}
\item Use the definition and that skill used in the previous exercises.
\begin{enumerate}
\item By definition we have 
\[\lag (a,b),T^*(x)\rag =\lag (2a+b,a-3b),(3,5)\rag =11a-12b.\]
We may observe that $T^*(x)=(11,-12)$.
\item By definition we have 
\[\lag (z_1,z_2),T^*(x)\rag =\lag (2z_1+iz_2,(1-i)z_1),(3-i,1+2i)\rag \]
\[=(5-i)z_1+(-1+3i)z_2.\]
We may observe that $T^*(x)=(5+i,-1-3i)$.
\item By definition we have 
\[\lag at+b,T^*(f)\rag =\lag 3at+(a+3b),4-2t\rag \]
\[=\int_{-1}^1{-6at^2+(10a-6b)t+(4a+12b)dt} = 4a + 24b.\]
By direct computation, we have  
\[\begin{aligned}
    \lag at + b, t\rag &= \frac{2}{3}a \text{ and } \\
    \lag at + b, 1\rag &= 2b.
\end{aligned}\]
We may observe that $T^*(f) = 6t+12$.
\end{enumerate}
\item \begin{description}
\item[(b)] Compute 
\[\lag x,(cT)^*(y)\rag =\lag cT(x),y\rag =\]
\[\lag T(x),\overline{c}y\rag =\lag x,T^*(\overline{c}y)\rag =\lag x,\overline{c}T^*(y)\rag \]
for all $x$ and $y$.
\item[(c)] Compute 
\[\lag x,(TU)^*(y)\rag =\lag TU(x),y\rag \]
\[=\lag U(x),T^*(y)\rag =\lag x,U^*T^*(y)\rag \]
for all $x$ and $y$.
\item[(e)] Compute 
\[\lag x,I^*(y)\rag =\lag I(x),y\rag \]
\[=\lag x,y\rag =\lag x,I(y)\rag \]
for all $x$ and $y$.
\end{description}
\item \begin{enumerate}
\item Just write it down as that in the proof of (c).
\begin{description}
\item[(a)] Compute 
\[L_{(A+B)^*}=L_{A+B}^*=(L_A+L_B)^*\]
\[=(L_A)^*+(L_B)^*=L_{A^*}+L_{B^*}=L_{A^*+B^*}.\]
\item[(b)] Compute 
\[L_{(cA)^*}=(L_{cA})^*=(cL_A)^*\]
\[=\overline{c}(L_A)^*=\overline{c}L_{A^*}=L_{\overline{c}A^*}.\]
\item[(d)] Compute 
\[L_{A^{**}}=(L_A)^{**}=L_A.\]
\item[(e)] Compute 
\[L_{I^*}=(L_I)^*=L_I.\]
\end{description}
\item The statement for nonsquare matrices has no difference with the Corollary.  These results come from that $A^*$ is the conjugate of $A^t$.
\end{enumerate}
\item Compute 
\[U_1^*=(T+T^*)^*=T^*+T^{**}=T^*+T=U_1\]
and 
\[U_2^*=(TT^*)^*=T^{**}T^*=TT^*=U_2.\]
\item Let $A=\begin{pmatrix}1&1\\0&0\end{pmatrix}$. Then $N(A)\neq N(A^*)$ since $(0,1)$ is an element only in the later one.
\item If $T$ is invertible, then we have the inverse mapping $T^{-1}$. Then we have 
\[
    T^*(T^{-1})^*=(T^{-1}T)^*=I^*=I
\]
and  
\[
    (T^{-1})^*T^*=(TT^{-1})^*=I^*=I.
\]
\item For each vector $v\in V$ we may write $v=v_1+v_2$ such that $v_1\in W$ and $v_2\in W\pp$. Now check 
\[\lag x_1+x_2,T^*(y_1+y_2)\rag =\lag T(x_1+x_2),y_1+y_2\rag \]
\[=\lag x_1,y_1+y_2\rag =\lag x_1,y_1 \rag \]
and 
\[\lag x_1+x_2,T(y_1+y_2)\rag =\lag x_1+x_2,y_1\rag \]
\[=\lag x_1,y_1\rag =\lag x_1+x_2,T^*(y_1+y_2)\rag \]
for all $x=x_1+x_2$ and $y=y_1+y_2$.
\item The sufficiency is easy since we may just pick $y=x$. For the necessity, suppose now $\|T(x)\|=\|x\|$. By Exercise 6.1.20 we have 
\[\lag x,y\rag =\frac{1}{4}\sum_{k=1}^4{i^k\|x+i^ky\|^2}\]
\[=\frac{1}{4}\sum_{k=1}^4{i^k\|T(x+i^ky)\|^2}=\frac{1}{4}\sum_{k=1}^4{i^k\|T(x)+i^kT(y)\|^2}\]
\[\lag T(x),T(y)\rag \]
if $\F=\mathbb{C}$. However, for the case $\F=\R$ the above argument could also work if we just pick $k$ to be $2$ and $4$.
\item We have 
\[0=\lag T^*T(x),x\rag =\lag T(x),T(x)\rag =\|T(x)\|^2\]
for all $x$ and hence $T(x)=0$ for all $x$. And the second statement is also true since we may write 
\[TT^*=T^{**}T^*=T_0\]
and get $T^*=T_0$. Since $T=T^{**}=T_0^*=T_0$.
\item \begin{enumerate}
\item If $x\in R(T^*)\pp$ we have 
\[0=\lag x,T^*(y)\rag =\lag T(x),y\rag \]
for all $y$. This means that $T(x)=0$ and so $x\in N(T)$. Conversely, if $x\in N(T)$, we have 
\[\lag x,T^*(y)\rag =\lag T(x),y\rag =0\]
for all $y$. This means that $x$ is an element in $R(T^*)\pp$.
\item By Exercise 6.2.13(c) we have 
\[N(T)\pp =(R(T^*)\pp )\pp =R(T^*)\pp .\]
\end{enumerate}
\item \begin{enumerate}
\item If $x\in N(T^*T)$ we have $T^*T(x)=0$ and 
\[0=\lag T^*T(x),x\rag =\lag T(x),T(x)\rag .\]
This means that $T(x)=0$ and $x\in N(T)$. Conversely, if $x\in N(T)$, we have $T^*T(x)=T^*(0)=0$ and so $x\in N(T^*T)$.

In summary, we have the nullities of $T^*T$ and $T$ are the same, and this implies the ranks of $T^*T$ and $T$ are the same by the fact that $V$ is finite dimensional and the dimension theorem.
%% On the other hand, since the dimension is finite, we have 
%% \[R(T^*T)=N(T^*T)\pp =N(T)\pp =R(T^*)\]
%% by the previous exercise. Hence we have  
%% \[\rank(T^*T)=\rank(T^*)=\rank(T)\]
%% by the next argument.
\item For arbitrary matrix $A$, denote $\overline{A}$ to be the matrix consisting of the conjugate of entris of $A$. Thus we have $A^*=\overline{A^t}$. We want to claim that $\rank(A)=\rank(A^*)$ first. Since we already have that $\rank(A)=\rank(A^t)$, it's sufficient to show that $\rank(A)=\rank(\overline{A})$. By Theorem 3.6 and its Corollaries, we may just prove that $\{v_i\}_{i\in I}$ is independent if and only if $\{\overline{v_i}\}_{i\in I}$ is independent, where $\overline{v_i}$ means the vector obtained from $v_i$ by taking conjugate to each coordinate. And it comes from the fact 
\[\sum_{i\in I}a_iv_i=0\]
if and only if 
\[\sum_{i\in I}\overline{a_i}\overline{v_i}=\overline{\sum_{i\in I}a_iv_i}=0.\]
Finally, by Theorem 6.10 we already know that $[T]_{\beta}^*=[T^*]_{\beta}$ for any orthonormal basis $\beta$. This means that $\rank(T)=\rank(T^*)$. And so 
\[\rank(TT^*)=\rank(T^{**}T^*)=\rank(T^*)=\rank(T).\]
%We define a mapping $T_c:R(L_A)\rightarrow R(L_{\overline{A}})$ by $T_c(Ax)=\overline{Ax}$. This is an isomorphism since $\overline{A}y$ is mapped by $A\overline{y}$ and $T(Ax)=\overline{Ax}=0$ implies $Ax=0$. So we have 
%\[\rank(A)=\rank(A^t)=\rank(\overline{A^t}).\]
\item It comes from the fact $L_{A^*}=(L_A)^*$.
\end{enumerate}
\item It's linear since 
\[T(cx_1+x_2)=\lag cx_1+x_2,y\rag z\]
\[=c\lag x_1,y\rag z+\lag x_2,y\rag z=cT(x_1)+T(x_2).\]
On the other hand, we have 
\[\lag u,T^*(v)\rag =\lag \lag u,y\rag z,v\rag \]
\[=\lag u,y\rag \lag z,v\rag =\lag u,\lag v,z\rag y\rag \]
for all $u$ and $v$. So we have 
\[T^*(x)=\lag x,z\rag y.\]
\item \begin{enumerate}
\item Let $y\in W$ be given. We may define 
\[g_y(x) =\lag T(x),y\rag_2,\] 
which is linear since $T$ is linear and the first component of inner product function is also linear. By Theorem 6.8 we may find an unique vector, called $T^*(y)$, such that 
\[\lag x,T^*(y)\rag_1 =\lag T(x),y\rag_2\]
for all $x$. This means $T^*(y)$ is always well-defined. It's unique since 
\[\lag x,T^*(y)\rag_1 =\lag x,U(y)\rag_1\]
for all $x$ and $y$ implies that $T^*=U$.

Finally, it's also linear since 
\[\lag x,T^*(y+cz)\rag_1 =\lag T(x),y+cz\rag_2 \]
\[=\lag T(x),y\rag_2 +\overline{c}\lag T(x),z\rag_2 \]
\[=\lag x,T^*(y)\rag_1 +\overline{c}\lag x,T^*(z)\rag_1 \]
\[=\lag x,T^*(y)\rag_1 +\lag x,cT^*(z)\rag_1 =\lag x,T^*(y)+cT^*(z)\lag_1 \]
for all $x$, $y$, and $z$.
\item Let 
\[\beta =\{v_1,v_2,\ldots ,v_m\}\]
and
\[\gamma =\{u_1,u_2,\ldots ,u_n\}.\]
Further, assume that 
\[T(v_j)=\sum_{i=1}^n{a_{ij}u_i}.\]
This means that $[T]_{\beta}^{\gamma}=\{a_{ij}\}$. 

On the other hand, assume that 
\[T^*(u_j)=\sum_{i=1}^n{c_{ij}v_i}.\]
And this means 
\[\overline{c_{ij}}=\lag v_i,T^*(u_j)\rag_1 =\lag T(v_i),u_j\rag_2 =a_{ji}\]
and $[T^*]_{\gamma}^{\beta}=([T]_{\beta}^{\gamma})^*$.
\item It comes from the same reason as Exercise 6.3.13(b).
\item See 
\[\lag T^*(x),y\rag =\overline{\lag y,T^*(x)\rag }\]
\[=\overline{\lag T(y),x\rag }=\lag x,T^*(y)\rag .\]
\item If $T(x)=0$ we have $T^*T(x)=T^*(0)=0$. If $T^*T(x)=0$ we have 
\[0=\lag x,T^*T(x)\rag =\lag T(x),T(x)\rag \]
and hence $T(x)=0$.
\end{enumerate}
\item \begin{enumerate}
\item Compute 
\[\lag x,(T+U)^*(y)\rag_1=\lag (T+U)(x),y\rag_2\]
\[=\lag T(x)+U(x),y\rag_2 =\lag T(x),y\rag_2+\lag U(x),y\rag_2 \]
\[=\lag x,T^*(y)\rag_1+\lag x,U^*(y)\rag_1=\lag x,(T^*+U^*)(y)\rag_1\]
for all $x$ and $y$.
\item Compute 
\[\lag x,(cT)^*(y)\rag_1 =\lag cT(x),y\rag_2 =\]
\[\lag T(x),\overline{c}y\rag_2 =\lag x,T^*(\overline{c}y)\rag_1 =\lag x,\overline{c}T^*(y)\rag_1 \]
for all $x$ and $y$.
\item Let $T$ is a mapping on $W$ and $U$ is a mapping from $V$ to $W$. Compute 
\[\lag x,(TU)^*(y)\rag_1 =\lag TU(x),y\rag_2 \]
\[=\lag U(x),T^*(y)\rag_2 =\lag x,U^*T^*(y)\rag_1 \]
for all $x$ and $y$.

Let $T$ is a mapping from $V$ to $W$ and $U$ is a mapping on $V$. Compute 
\[\lag x,(TU)^*(y)\rag_1 =\lag TU(x),y\rag_2 \]
\[=\lag U(x),T^*(y)\rag_1 =\lag x,U^*T^*(y)\rag_1 \]
for all $x$ and $y$.
\item Compute 
\[\lag x,T^{**}(y)\rag_2 =\lag T^*(x),y\rag_1 \]
\[=\lag x,T(y)\rag \]
for all $x$ and $y$.
\end{enumerate}
\item If $x\in R(T^*)\pp$ we have 
\[0=\lag x,T^*(y)\rag_1 =\lag T(x),y\rag_2 \]
for all $y$. This means that $T(x)=0$ and so $x\in N(T)$. Conversely, if $x\in N(T)$, we have 
\[\lag x,T^*(y)\rag_1 =\lag T(x),y\rag_2 =0\]
for all $y$. This means that $x$ is an element in $R(T^*)\pp$.
\item For arbitrary matrix $M$ we already have $\det(M)=\det(M^t)$. So it's sufficient to show that $\overline{\det(M)}=\det(\overline{M})$. We prove this by induction on $n$, the size of a matrix. For $n=1$, we have 
\[\overline{\det\begin{pmatrix}a\end{pmatrix}}=\det\begin{pmatrix}\overline{a}\end{pmatrix}.\]
For $n=2$, we have 
\[\overline{\det\begin{pmatrix}a&b\\c&d\end{pmatrix}}=\overline{ad-bc}\]
\[=\overline{a}\overline{d}-\overline{b}\overline{c}=\det\begin{pmatrix}\overline{a}&\overline{b}\\\overline{c}&\overline{d}\end{pmatrix}.\]
Suppose the hypothesis is true for $n=k-1$. Consider a $k\times k$ matrix $M$. We have 
\[\overline{\det(M)}=\overline{\sum_{j=1}^k{(-1)^{i+j}M_{ij}\cdot \det(\tilde{M}_{ij})}}\]
\[=\sum_{j=1}^k{(-1)^{i+j}\overline{M_{ij}}\cdot \overline{\det(\tilde{M}_{ij})}}\]
\[=\sum_{j=1}^k{(-1)^{i+j}\overline{M_{ij}}\cdot \det(\tilde{\overline{M}}_{ij})}=\det(\overline{M}).\]
This means that 
\[\det(A)=\det(A^t)=\det(A^*).\]
\item Let $v_i$ be the $i$-th column of $A$. Then we have $v_i^*$ is the $i$-th row of $A^*$. And the desired result comes from the fact
\[(A^*A)_{ij}=v_i^*v_j=\lag v_j,v_i\rag , \]
which is zero when $i\neq j$.
\item Follow the method after Theorem 6.12.
\begin{enumerate}
\item The linear function is $-2t+\frac{5}{2}$ with error $E=1$. The quadratic function is $\frac{1}{3}t^2-\frac{4}{3}t+2$ with the error $E=0$.
\item The linear function is $\frac{5}{4}t+\frac{11}{20}$ with error $E=\frac{3}{10}$. The quadratic function is $\frac{1}{56}t^2+\frac{15}{14}+\frac{239}{280}$ with the error $E=\frac{8}{35}$.
\item The linear function is $-\frac{9}{5}t+\frac{4}{5}$ with error $E=\frac{2}{5}$. The quadratic function is $-\frac{1}{7}t^2-\frac{9}{5}t+\frac{38}{35}$ with the error $E=\frac{4}{35}$.
\end{enumerate}
\item Follow the same method. We have the linear function is $2.1x-\frac{127}{20}$. So the spring constant is $2.1$.
\item As the statement in Theorem 6.13, we may first find a vector $u$ such that $AA^*u=b$. Finally the minimal solution would be $A^*u$.
\begin{enumerate}
\item The minimal solution is $(2,4,-2)$.
\item The minimal solution is $\frac{1}{7}(2,3,1)$.
\item The minimal solution is $(1,-\frac{1}{2},\frac{1}{2})$.
\item The minimal solution is $\frac{1}{12}(7,1,3,-1)$.
\end{enumerate}
\item \begin{enumerate}
\item Direct calculate that 
\[A^*A=\begin{pmatrix}\sum_{i=1}^m{t_i^2} &\sum_{i=1}^m{t_i}\\\sum_{i=1}^m{t_i} & m\end{pmatrix}\]
and get the result by 
\[A^*A\begin{pmatrix}c\\d\end{pmatrix}=A^*y.\]
For the second method, we may calculate that 
\[E=\sum_{i=1}^m{(y_i-ct_i-d)^2}\]
and $\frac{\partial E}{\partial c}=0$ and $\frac{\partial E}{\partial d}=0$ give the normal equations.
\item We want to claim that $\overline{y}=c\overline{t}+d$, whrer $\overline{t}$, $\overline{y}$ are defined in the question and $c$, $d$ are a solution of the normal equations. But this is an instant result by dividing the second equation by $m$.
\end{enumerate}
\item \begin{enumerate}
\item Check 
\[T(c\sigma +\tau)(k)=\sum_{i=k}^{\infty}{(c\sigma+\tau)(k)}\]
\[=c\sum_{i=k}^{\infty}{(\sigma)(k)}+\sum_{i=k}^{\infty}{(\tau)(k)}\]
\[=cT(\sigma)(k)+T(\tau)(k).\]
\item For $k\leq n$ we have 
\[T(e_n)(k)=\sum_{i=k}^{\infty}{e_n(i)}=1=\sum_{i=1}^n{e_i}(k).\]
And for $k> n$ we have 
\[T(e_n)(k)=\sum_{i=k}^{\infty}{e_n(i)}=0=\sum_{i=1}^n{e_i}(k).\]
\item Suppoe that $T^*$ exist. We try to compute $T^*(e_1)$ by 
\[1\cdot \overline{T^*(e_1)}(i)=\lag e_i,T^*(e_1)\rag =\lag \sum_{i=1}^n{e_i},e_1\rag =1.\]
This means that $T^*(e_1)(i)=1$ for all $i$. This is impossible since $T^*(e_1)$ is not an element in $V$.
\end{enumerate}
\end{enumerate}
