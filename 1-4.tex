\section{Linear Combinations and Systems of Linear Equations}
\begin{enumerate}
\item \begin{enumerate}
\item Yes. Just pick any coeficient to be zero.
\item No. By definition it should be $\{0\}$.
\item Yes. Every subspaces of which $S$ is a subset contains span$(S)$ and span$(S)$ is a subspace.
\item No. This action will change the solution of one system of linear equations.
\item Yes.
\item No. For example, $0x=3$ has no solution.
\end{enumerate}
\item \begin{enumerate}
\item Original system $\Leftrightarrow \left\{ \begin{array}{ccccccccc}x_1&-&x_2&-&2x_3&-&x_4&=&-3\\&&&&x_3&+&2x_4&=&4\\&&&&4x_3&+&8x_4&=&16\end{array}\right.$. So we have solution is $\{(5+s-3t,s,4-2t,t):s,t\in \mathbb{F}\}$.
\item $\{(-2,-4,-3)\}$.
\item No solution.
\item $\{(-16-8s,9+3s,s,2):s\in \mathbb{F}\}$.
\item $\{(-4+10s-3t,3-3s+2t,r,s,5):s,t \in \mathbb{F}\}$.
\item $\{(3,4,-2)\}$.
\end{enumerate}
\item \begin{enumerate}
\item Yes. Solve the equation $x_1(1,3,0)+x_2(2,4,-1)=(-2,0,3)$ and we have the solution $(x_1,x_2)=(4,-3)$.
\item Yes.
\item No.
\item No.
\item No.
\item Yes.
\end{enumerate}
\item \begin{enumerate}
\item Yes.
\item No.
\item Yes.
\item Yes.
\item No.
\item No.
\end{enumerate}
\item \begin{enumerate}
\item Yes.
\item No.
\item No.
\item Yes.
\item Yes.
\item No.
\item Yes.
\item No.
\end{enumerate}
\item For every $(x_1,x_2,x_3)\in \mathbb{F}^3$ we may assume \[y_1(1,1,0)+y_2(1,0,1)+y_3(0,1,1)=(x_1,x_2,x_3)\] and solve the system of linear equation. We got $(x_1,x_2,x_3)=\frac{1}{2}(x_1-x_2+x_3)(1,1,0)+\frac{1}{2}(x_1+x_2-x_3)(1,0,1)+\frac{1}{2}(-x_1+x_2+x_3)(0,1,1)$.
\item For every $(x_1,x_2,\dots x_n)\in \mathbb{F}^n$ we can write $(x_1,x_2,\dots x_n)=x_1e_1+x_2e_2+\cdots +x_ne_n$.
\item It's similar to exercise 1.4.7.
\item It's similar to exercise 1.4.7.
\item For $x\neq 0$ the statement is the definition of linear combination and the set is a line. For $x=0$ the both side of the equation is the set of zero vector and the set is the origin.
\item To prove it's sufficient we can use Theorem 1.5 and then we know $W=\mathrm{span}(W)$ is a subspace. To prove it's necessary we can also use Theorem 1.5. Since $W$ is a subspace contains $W$, we have $\mathrm{span}(W)\subset W$. On the other hand, it's natural that $\mathrm{span}(W)\supset W$.
\item To prove $\mathrm{span}(S_1)\subset \mathrm{span}(S_2)$ we may let $v\in S_1$. Then we can write $v=a_1x_1+a_2x_2+\cdots +a_3x_3$ where $x_i$ is an element of $S_1$ and so is $S_2$ for all $n=1,2,\dots,n$. But this means $v$ is a linear combination of $S_2$ and we complete the proof. If $\mathrm{span}(S_1)=V$, we know $\mathrm{span}(S_2)$ is a subspace containing $\mathrm{span}(S_1)$. So it must be $V$.
\item We prove $\mathrm{span}(S_1\cup S_2)\subset \mathrm{span}(S_1)+\mathrm{span}(S_2)$ first. For $v\in \mathrm{span}(S_1\cup S_2)$ we have $v=\sum_{i=1}^n{a_ix_i}+\sum_{j=1}^m{b_jy_j}$ with $x_i\in S_1$ and $y_j \in S_2$. Since the first summation is in $\mathrm{span}(S_1)$ and the second summation is in $\mathrm{span}(S_2)$, we have $v\in \mathrm{span}(S_1)+\mathrm{span}(S_2)$. For the converse, let $u+v\in \mathrm{span}(S_1)+\mathrm{span}(S_2)$ with $u\in \mathrm{span}(S_1)$ and $v\in \mathrm{span}(S_2)$. We can right $u+v=\sum_{i=1}^n{a_ix_i}+\sum_{j=1}^m{b_jy_j}$ with $x_i\in S_1$ and $y_j \in S_2$ and this means $u+v\in \mathrm{span}(S_1\cup S_2)$.
\item For $v\in \mathrm{span}(S_1\cap S_2)$ we may write $v=\sum_{i=1}^n{a_ix_i}$ with $x_i \in S_1$ and $x_i \in S_2$. So $v$ is an element of both $\mathrm{span}(S_1)$ and $\mathrm{span}(S_2)$ and hence an element of $\mathrm{span}(S_1)\cap \mathrm{span}(S_2)$. For example we have if $S_1=S_2=(1,0)$ then they are the same and if $S_1=(1,0)$ and $S_2=(0,1)$ then we have the left hand side is the set of zero vector and the right hand side is the the plane $\mathbb{R}^2$.
\item If we have both $a_1v_1+a_2v_2+\cdots +a_nv_n=b_1v_1+b_2v_2+\cdots b_nv_n$ then we have $(a_1-b_1)v_1+(a_2-b_2)v_2+\cdots +(a_n-b_n)v_n=0$. By the property we can deduce that $a_i=b_i$ for all $i$.
\item When $W$ has finite element the statement holds. Otherwise $W-\{v\}$, where $v\in W$ will be a generating set of $W$. But there are infinitely many $v\in W$.
\end{enumerate}