\section{The Jordan Canonical Form I}
\begin{enumerate}
\item \begin{enumerate}
\item Yes. It comes directly from the definition.
\item No. If $x$ is a generalized eigenvector, we can find the smallest positive integer $p$ such that $(T-\lambda I)^p(x)=0$. Thus $y=(T-\lambda I)^{p-1}\neq 0$ is an eigenvector with respect to the eigenvalue $\lambda$. Hence $\lambda $ must be an eigenvalue.
\item No. To apply the theorems in this section, the \charpoly{} should split. For example, the matrix $\begin{pmatrix}0&-1\\1&0\end{pmatrix}$ over $\R$ has no eigenvalues.
\item Yes. This is a result of Theorem 7.6.
\item No. The identity mapping $I_2$ from $\C^2$ to $\C^2$ has two cycles for the eigenvalue $1$.
\item No. The basis $\beta_i$ may not consisting of a union of cycles. For example, the transformation $T(a,b)=(a+b,b)$ has only one eigenvalue $1$. The generalized eigenspace $K_1=\F^2$. If 
\[\beta=\{(1,1),(1,-1)\},\]
then the matrix representation would be 
\[[T]_{\beta}=\frac{1}{2}\begin{pmatrix}3&-1\\1&1\end{pmatrix},\]
which is not a Jordan form.
\item Yes. Let $\alpha $ be the standard basis. Then $[L_J]_{\alpha}=J$ is a Jordan form.
\item Yes. This is Theorem 7.2.
\end{enumerate}
\item Compute the \charpoly{} to find eigenvalues as what we did before. For each $\lambda$, find a basis for $K_{\lambda}$ consisting of a union of disjoint cycles by computing bases for the null space of $(A-\lambda I)^p$ for each $p$. Write down the matrix $S$ whose columns consist of these cycles of generalized eigenvectors. Then we will get the Jordan canonical form $J=S^{-1}AS$. When the matrix is diagonalizable, the Jordan canonical form should be the diagonal matrix sismilar to $A$. For more detail, please see the examples in the textbook. On the other hand, these results were computed by Wolfram Alpha. For example, the final question need the command below.
\begin{verbatim}
JordanDecomposition[{{2,1,0,0},{0,2,1,0},{0,0,3,0},{0,1,-1,3}}]
\end{verbatim}
\begin{enumerate}
\item 
\[S=\begin{pmatrix}1&-1\\1&0\end{pmatrix}, J=\begin{pmatrix}2&1\\0&2\end{pmatrix}.\]
\item 
\[S=\begin{pmatrix}-1&2\\1&3\end{pmatrix}, J=\begin{pmatrix}-1&0\\0&4\end{pmatrix}.\]
\item 
\[S=\begin{pmatrix}1&1&1\\3&1&2\\0&1&0\end{pmatrix}, J=\begin{pmatrix}-1&0&0\\0&2&1\\0&0&2\end{pmatrix}.\]
\item 
\[S=\begin{pmatrix}1&0&0&1\\0&1&0&1\\0&0&0&1\\0&-1&1&0\end{pmatrix}, J=\begin{pmatrix}2&1&0&0\\0&2&0&0\\0&0&3&0\\0&0&0&3\end{pmatrix}.\]
\end{enumerate}
\item Pick one basis $\beta $ and write down the matrix representation $[T]_{\beta}$. Then do the same thing in the previous exercises. Again, we denote the Jordan canonical form by $J$ and the matrix consisting of Jordan canonical basis by $S$. The Jordan canonical basis is the set of vector in $V$ corresponding those column vectors of $S$ in $\F^n$.
\begin{enumerate}
\item Pick $\beta$ to be the standard basis 
\[\{1,x,x^2\}\]
and get 
\[[T]_{\beta}=\begin{pmatrix}-2&-1&0\\0&2&-2\\0&0&2\end{pmatrix}\]
and 
\[S=\begin{pmatrix}1&-1&\frac{1}{4}\\0&4&0\\0&0&-2\end{pmatrix}, J=\begin{pmatrix}-2&0&0\\0&2&1\\0&0&2\end{pmatrix}.\]
\item Pick $\beta$ to be the basis 
\[\{1,t,t^2,e^t,te^t\}\]
and get 
\[[T]_{\beta}=\begin{pmatrix}0&1&0&0&0\\0&0&2&0&0\\0&0&0&0&0\\0&0&0&1&1\\0&0&0&0&1\end{pmatrix}\]
and 
\[S=\begin{pmatrix}1&0&0&0&0\\0&1&0&0&0\\0&0&\frac{1}{2}&0&0\\0&0&0&1&0\\0&0&0&0&1\end{pmatrix}, J=\begin{pmatrix}0&1&0&0&0\\0&0&1&0&0\\0&0&0&0&0\\0&0&0&1&1\\0&0&0&0&1\end{pmatrix}.\]
\item Pick $\beta$ to be the standard basis 
\[\{\begin{pmatrix}1&0\\0&0\end{pmatrix},\begin{pmatrix}0&1\\0&0\end{pmatrix},\begin{pmatrix}0&0\\1&0\end{pmatrix},\begin{pmatrix}0&0\\0&1\end{pmatrix}\}\]
and get 
\[[T]_{\beta}=\begin{pmatrix}1&0&1&0\\0&1&0&1\\0&0&1&0\\0&0&0&1\end{pmatrix}\]
and 
\[S=\begin{pmatrix}1&0&0&0\\0&0&1&0\\0&1&0&0\\0&0&0&1\end{pmatrix}, J=\begin{pmatrix}1&1&0&0\\0&1&0&0\\0&0&1&1\\0&0&0&1\end{pmatrix}.\]
\item Pick $\beta$ to be the standard basis 
\[\{\begin{pmatrix}1&0\\0&0\end{pmatrix},\begin{pmatrix}0&1\\0&0\end{pmatrix},\begin{pmatrix}0&0\\1&0\end{pmatrix},\begin{pmatrix}0&0\\0&1\end{pmatrix}\}\]
and get 
\[[T]_{\beta}=\begin{pmatrix}3&0&0&0\\0&2&1&0\\0&1&2&0\\0&0&0&3\end{pmatrix}\]
and 
\[S=\begin{pmatrix}1&0&0&0\\0&0&0&1\\0&1&1&0\\0&-1&1&0\end{pmatrix}, J=\begin{pmatrix}3&0&0&0\\0&3&0&0\\0&0&3&0\\0&0&0&1\end{pmatrix}.\]
\end{enumerate}
\item We may observe that $W=\sp(\gamma)$ is $(T-\lambda I)$-invariant by the definition of a cycle. Thus for all $w\in W$, we have 
\[T(w)=(T-\lambda I)(w)+\lambda I(w)=(T-\lambda I)(w)+\lambda w\in W.\]
\item If $x$ is an element of in two cycles, which is said to be $\gamma_1$ and $\gamma_2$ without lose of generality, we may find the smallest ingeter $q$ such that 
\[(T-\lambda I)^q(x)=0.\]
This means that the initial eigenvectors of $\gamma_1$ and $\gamma_2$ are both $(T-\lambda I)^{q-1}(x)$. This is a contradiction. Hence all cycles are disjoint.
\item \begin{enumerate}
\item Use the fact that $T(x)=0$ if only if $(-T)(x)=-0=0$.
\item Use the fact that $(-T)^k=(-1)^kT$.
\item It comes from the fact 
\[(\lambda I_V-T)^k=[-(T-\lambda I_V)]^k\] 
and the previous argument.
\end{enumerate}
\item \begin{enumerate}
\item If $U^k(x)=0$, then $U^{k+1}(x)=U^k(U(x))=0$.
\item We know $U^{m+1}(V)=U^m(U(V))\subset U^m(V)$. With the assumption $\rank(U^m)=\rank(U^{m+1})$, we know that 
\[U^{m+1}(V)=U^m(V).\]
This means 
\[U(U^m(V))=U^m(V)\]
and so $U^k(V)=U^m(V)$ for all integer $k\geq m$.
\item The assumption $\rank(U^m)=\rank(U^{m+1})$ implies 
\[\nul(U^m)=\nul(U^{m+1})\]
by Dimension Theorem. This means $N(U^m)=N(U^{m+1})$ by the previous argument. If $U^{m+2}(x)=0$, then $U(x)$ is an element in $N(U^{m+1})=N(U^m)$. Hence we have $U^m(U(x))=0$ and thus $x$ is an element in $N(U^{m+1})$. This means that $N(U^{m+2})\subset N(U^{m+1})$ and so they are actually the same. Doing this inductively, we know that $N(U^m)=N(U^k)$ for all integer $k\geq m$.
\item By the definition of $K_{\lambda}$, we know 
\[K_{\lambda}=\cup_{p\geq 1}N((T-\lambda I)^p).\]
But by the previous argument we know that 
\[N((T-\lambda I)^m)=N((T-\lambda I)^k)\]
for all integer $k\geq m$ and the set is increasing as $k$ increases. So actually $K_{\lambda}$ is $N((T-\lambda I)^m)$.
\item Since the \charpoly{} splits, the transformation $T$ is diagonalizable if and only if $K_{\lambda}=E_{\lambda}$. By the previous argument, we know that 
\[K_{\lambda}=N(T-\lambda I)=E_{\lambda}.\]
\item If $\lambda $ is an eigenvalue of $T_W$, then $\lambda $ is also an eigenvalue of $T$ by Theorem 5.21. Since $T$ is diagonalizable, we have the condition 
\[\rank(T-\lambda I)=\rank((T-\lambda I)^2)\]
and so 
\[N(T-\lambda I)=N((T-\lambda I)^2)\]
by the previous arguments. This implies 
\[N(T_W-\lambda I_W)=N((T_W-\lambda I_W)^2).\]
By Dimension Theorem, we get that 
\[\rank(T_W-\lambda I_W)=\rank((T_W-\lambda I_W)^2).\]
So $T_W$ is diagonalizable.
\end{enumerate}
\item Theorem 7.4 implies that 
\[V=\oplus_{\lambda}K_{\lambda}.\]
So the representation is uniques.
\item \begin{enumerate}
\item The subspace $W$ is $T$-invariant by Exercise 7.1.4. Let $\{v_i\}_i$ be the cycle with initial vector $v_1$. Then $[T_W]_{\gamma}$ is a Jordan block by the fact 
\[T_W(v_1)=\lambda v_1\]
and 
\[T_W(v_i)=v_{i-1}+\lambda v_i\]
for all $i>1$. And $\beta $ is a Jordan canonical basis since each cycle forms a block.
\item If the $ii$-entry of $[T]_{\beta}$ is $\lambda$, then $v_i$ is an nonzero element in $K_{\lambda}$. Since $K_{\lambda}\cap K_{\mu}=\{0\}$ for distinct eigenvalues $\lambda $ and $\mu$, we know that $\beta'$ is exactly those $v_i$'s such that the $ii$-entry of $[T]_{\beta}$ is $\lambda$. Let $m$ be the number of the eigenvalue $\lambda$ in the diagonal of $[T]_{\beta}$. Since a Jordan form is upper triangular. We know that $m$ is the multiplicity of $\lambda$. By Theorem 7.4(c) we have 
\[\dim(K_{\lambda})=m=|\beta'|.\]
So $\beta'$ is a basis for $K_{\lambda}$.
\end{enumerate}
\item \begin{enumerate}
\item Those initial vectors of $q$ disjoint cycles forms a independent set consisting of eigenvectors in $E_{\lambda}$.
\item Each block will correspond to one cycle. Use the previous argument and get the result.
\end{enumerate}
\item By Theorem 7.7 and the assumptiong that the \charpoly{} of $L_A$ splits, the transformation $L_A$ has a Jordan form $J$ and a corresponding Jordan canonical basis $\beta$. Let $\alpha$ be the standarad basis for $\F^n$. Then we have 
\[J=[L_A]_{\beta}=[I]_{\alpha}^{\beta}A[I]_{\beta}^{\alpha}\]
and so $A$ is similar to $J$.
\item By Theorem 7.4(b), the space $V$ is the direct sum of each $\sp(\beta_i)=K_{\lambda_i}$, where $\beta_i$ is a basis for $K_{\lambda_i}$.
\item Denote $K_{\lambda_i}$ by $W_i$. Let $\beta_i$ be a Jordan canonical basis for $T_{W_i}$. Apply Theorem 5.25 and get the result.
\end{enumerate}