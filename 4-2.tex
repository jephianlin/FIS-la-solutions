\section{Determinants of Order $n$}
\begin{enumerate}
\item \begin{enumerate}
\item No. See Exercise 4.1.1(a).
\item Yes. This is Theorem 4.4.
\item Yes. This is the Corollary after Theorem 4.4.
\item Yes. This is Theorem 4.5.
\item No. For example, the determinant of $\begin{pmatrix}2&0\\0&1\end{pmatrix}$ is $2$ but not $\det(I)=1$.
\item No. We have that $\begin{pmatrix}1&0\\2&1\end{pmatrix}=1\neq 2\det(I)=2$. 
\item No. For example, the determinant of identity matrix is $1$.
\item Yes. See Exercise 4.2.23.
\end{enumerate}
\item Determinant is linear when we fixed all but one row. So we have that 
\[\det\begin{pmatrix}3a_1&3a_2&3a_3\\3b_1&3b_2&3b_3\\3c_1&3c_2&3c_3\end{pmatrix}=3\det\begin{pmatrix}a_1&a_2&a_3\\3b_1&3b_2&3b_3\\3c_1&3c_2&3c_3\end{pmatrix}\]
\[=9\det\begin{pmatrix}a_1&a_2&a_3\\b_1&b_2&b_3\\3c_1&3c_2&3c_3\end{pmatrix}=27\det\begin{pmatrix}a_1&a_2&a_3\\b_1&b_2&b_3\\c_1&c_2&c_3\end{pmatrix}.\]
Hene we know that $k=27$.
\item We can add $-\frac{5}{7}$ times  of the third row the the second row without changing the value of determinant and do the same as the previous exercise and get the conclusion that $k=2\times 3\times 7=42$.
\item See the following process.
\[\det\begin{pmatrix}b_1+c_1&b_2+c_2&b_3+c_3\\a_1+c_1&a_2+c_2&a_3+c_3\\a_1+b_1&a_2+b_2&a_3+b_3\end{pmatrix}=-\det\begin{pmatrix}-(b_1+c_1)&-(b_2+c_2)&-(b_3+c_3)\\a_1+c_1&a_2+c_2&a_3+c_3\\a_1+b_1&a_2+b_2&a_3+b_3\end{pmatrix}\]
\[=-2\det\begin{pmatrix}a_1&a_2&a_3\\a_1+c_1&a_2+c_2&a_3+c_3\\a_1+b_1&a_2+b_2&a_3+b_3\end{pmatrix}=-2\det\begin{pmatrix}a_1&a_2&a_3\\c_1&c_2&c_3\\b_1&b_2&b_3\end{pmatrix}\]
\[=2\det\begin{pmatrix}a_1&a_2&a_3\\b_1&b_2&b_3\\c_1&c_2&c_3\end{pmatrix}\]
The first equality comes from adding one time the second row and one time the third row to the first column and the second equality comes from adding $-1$ time the first row to the second and the third row. Finally we interchange the second and the third row and multiply the determinant by $-1$. Hence $k$ would be $2$.
\item The determinant should be $-12$ by following processes.
\[\det\begin{pmatrix}0&1&2\\-1&0&-3\\2&3&0\end{pmatrix}\]
\[=0\det\begin{pmatrix}0&-3\\3&0\end{pmatrix}-1\det\begin{pmatrix}-1&-3\\2&0\end{pmatrix}+2\det\begin{pmatrix}-1&0\\2&3\end{pmatrix}\]
\[=0\times 9-1\times 6+2\times (-3)=-12\]
\item The determinant should be $-13$.
\item The determinant should be $-12$.
\item The determinant should be $-13$.
\item The determinant should be $22$.
\item The determinant should be $4+2i$.
\item The determinant should be $-3$.
\item The determinant should be $154$.
\item The determinant should be $-8$.
\item The determinant should be $-168$.
\item The determinant should be $0$.
\item The determinant should be $36$.
\item The determinant should be $-49$.
\item The determinant should be $10$.
\item The determinant should be $-28-i$.
\item The determinant should be $17-3i$.
\item The determinant should be $95$.
\item The determinant should be $-100$.
\item Use induction on $n$, the size of the matrix. For $n=1$, every $1\times 1$ matrix is upper triangular and we have the fact $\det\begin{pmatrix}a\end{pmatrix}=a$. Assuming the statement of this exercise holds for $n=k$, consider any $(n+1)\times (n+1)$ upper triangular matrix $A$. We can expand $A$ along the first row with the formula
\[\det(A)=\sum_{j=1}^{n+1}{(-1)^{1+j}A_{1j}\det(\tilde{A}_{1j})}.\]
And the matrix $\tilde{A}_{1j}$, $j\neq 1$, contains one zero collumn and hence has rank less than $n+1$. By the Corollary after Theorem 4.6 those matrix has determinant $0$. However, we have the matrix $\tilde{A}_{11}$ is upper triangular and by induction hypothesis we have 
\[\det(\tilde{A}_{11})=\prod_{i=2}^{n+1}{A_{ii}}.\]
So we know the original formula would be 
\[\det(A)=A_{11}\det(\tilde{A}_{11})=\prod_{i=1}^{n+1}{A_{ii}}.\]
\item Let $z$ be the zero row vector. Thus we have that 
\[\det\begin{pmatrix}a_1\\a_2\\\vdots\\a_{r-1}\\z\\a_{r+1}\\\vdots\\a_n\end{pmatrix}=\det\begin{pmatrix}a_1\\a_2\\\vdots\\a_{r-1}\\0z\\a_{r+1}\\\vdots\\a_n\end{pmatrix}=0\det\begin{pmatrix}a_1\\a_2\\\vdots\\a_{r-1}\\z\\a_{r+1}\\\vdots\\a_n\end{pmatrix}=0\]
\item Applies Theorem 4.3 to each row. Thus we have hat 
\[\det\begin{pmatrix}ka_1\\ka_2\\\vdots\\ka_n\end{pmatrix}=k\det\begin{pmatrix}a_1\\ka_2\\\vdots\\ka_n\end{pmatrix}=\cdots =k^n\det\begin{pmatrix}a_1\\a_2\\\vdots\\a_n\end{pmatrix}.\]
\item By the previous exercise the equality holds only when $n$ is even or $\mathbb{F}$ has characteristic $2$.
\item If $A$ has two identical columns, then the matrix $A$ is not full-rank. By Corollary after Theorem 4.6 we know the determinant of $A$ should be $0$.
\item The matrix $E_1$ can be obtained from $I$ by interchanging two rows. So by Theorem 4.5 the determinant should be $-1$. The matrix $E_2$ is upper triangular. By Exercise 4.2.23 the determinant should be $c$, the scalar by whom some row was multiplied. The matrix $E_3$ has determinant $1$ by Theorem 4.6.
\item The elementary matrix of type $1$ and type $2$ is symmetric. So the statement holds naturally. Let $E$ be the elementary matrix of type $3$ of adding $k$ times of the $i$-th row to the $j$-th row. We know that $E^t$ is also an elementary matrix of type $3$. By the previous exercise we know that this kind of matrix must have determinant $1$.
\item We can interchange the $i$-th row and the $(n+1-i)$-th row for all $i=1,2,\ldots \lfloor \frac{n}{2}\rfloor$\footnote{The symbol $\lfloor x\rfloor$ means the greatest integer $m$ such that $m\leq x$.}. Each process contribute $-1$ one time. So we have that 
\[\det(B)=(-1)^{\lfloor \frac{n}{2}\rfloor}\det(A).\]
\end{enumerate}